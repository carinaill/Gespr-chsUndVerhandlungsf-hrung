\vspace{3\baselineskip}
\chapter{Aktives Zuhören}
Das aktive Zuhören ist "eine Kommunikationstechnik mit dem Ziel, eine 
positive Kommunikationsatmosphäre zu schaffen"\footnote{\cite{krebsjung2016}, S.44.}. Das geschieht mit Hilfe verbaler Signale,
wie zustimmende "hm" Geräusche oder Nachfragen wie "Ach echt?" oder "Und was ist dann passiert?"
sowie nonverbaler Signale wie Nicken, Körperhaltung und Augenkontakt.\footnote{Vgl. \cite{krebsjung2016}, S.44.}

\section{a: Positive Effekte bei bewusster Anwendung in Verhandlungssituationen}
Die bewusste Anwendung des aktiven Zuhörens kann in Verhandlungssituationen verschiedene 
positive Effekte auf den Verhandlungspartner haben. Der Gesprächspartner fühlt sich ernst genommen, verstanden 
und respektiert. Dies kann dazu beitragen, Unsicherheiten oder Anspannungen abzubauen 
und eine offenere Gesprächsatmosphäre zu schaffen.\\
Ein positives Verhandlungsklima wirkt sich insbesondere auf die Beziehungsebene aus. 
Wenn der Verhandlungspartner das Gefühl hat, dass seine Anliegen gehört werden, 
ist er häufig eher bereit, kooperativ zu handeln, eigene Positionen zu erklären 
oder Kompromisse einzugehen. Darüber hinaus kann aktives Zuhören langfristig das Vertrauen 
stärken, was dazu führen kann, dass der Verhandlungspartner auch bei zukünftigen 
Verhandlungen bevorzugt erneut den Kontakt sucht. Insgesamt unterstützt aktives Zuhören 
somit sowohl den kurzfristigen Verhandlungsverlauf als auch den Aufbau einer stabilen 
Geschäftsbeziehung.

\section{b: Elemente für die Gesprächsführung}
Einzelne Elemente des aktiven Zuhörens können nicht nur beziehungsfördernd, 
sondern auch strategisch und strukturierend in der Gesprächsführung eingesetzt werden. 
Nonverbale Signale wie Nicken und Blickkontakt vermitteln Zustimmung und Aufmerksamkeit 
und können den Gesprächspartner dazu ermutigen, seine Argumente weiter auszuführen. 
Dies ist besonders hilfreich, um zusätzliche Informationen zu erhalten, ohne den 
Gesprächsfluss zu unterbrechen.\\
Gezielte Nachfragen wie „Habe ich Sie richtig verstanden?“ oder „Können Sie das näher erläutern?“ 
dienen dazu, Inhalte zu präzisieren und Missverständnisse zu vermeiden. Gleichzeitig signalisieren 
sie Interesse und fördern die Beziehungsebene. Offene Fragen wie „Und wie ging es dann weiter?“ 
können strategisch genutzt werden, um das Gespräch zu lenken, neue Aspekte zu erschließen 
oder Zeit zu gewinnen, um eigene Argumente vorzubereiten. 