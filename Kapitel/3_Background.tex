\chapter{Background}

\counterwithin{equation}{chapter}


\(1_n = (1 \ 1 \ ... \ 1)^T\) is defined the one column vector, \(I_n\) is an \(n \times n\) identity matrix. \(\|a\|_2\) is the euclidean norm of a column vector \(a\) calculated as:

\begin{equation}
    \|a\|_2 = \sqrt{\sum_{i}^{n} a_i^2}
    \label{eqn:co_init1}
\end{equation}

It is defined as:

\begin{equation}
    D = (d_1 \ d_2 ... d_m) \ with \ d = (d_1 \ d_2 ... d_n)^T
    \label{eqn:co_init2}
\end{equation}

\begin{equation}
    M = (m_1 \ m_2 ... m_q)
    \label{eqn:co_init3}
\end{equation}

\begin{equation}
    Q = (q_1 \ q_2 ... q_k) \ with \ M \cap Q = \emptyset
    \label{eqn:co_init4}
\end{equation}

Their corresponding projections are \(P_m\) respectively \(P_q\).

With \(D'\) being all data records that are neither sticky nor shift-able, our dataset \(D\) is composed of the following components:

\begin{equation}
    \overbrace{\underbrace{D}_{n \times m}}^{Data} = \underbrace{D'}_{n \times (m-q-k)} \cup \overbrace{\underbrace{M}_{n \times q}}^{Shift\text{-}able \ Data} \cup \overbrace{\underbrace{Q}_{n \times k}}^{Sticky \ Data}
    \label{eqn:co_init5}
\end{equation}

Thus, \(M_\varepsilon\) equals \(M\).
 are defined by two polynomials \({f_1}^\delta\) and \({f_2}^\delta\) of degree 2 (see \ref{eqn:co_init6}).

\begin{equation}
    {f_1}^\delta = 3 \cdot \delta - 2\delta^2, \ {f_2}^\delta=1-4\cdot\delta + 4\delta^2
    \label{eqn:co_init6}
\end{equation}


... as shown in Formula (\ref{eqn:co1}).

\begin{equation}
    \overline{A}=\left( \frac{1}{1+\|m\|^2_2}\right)\Delta p \cdot m^T + A
    \label{eqn:co1}
\end{equation}

... shift-able sets is calculated by Formula (\ref{eqn:co2}) and Formula (\ref{eqn:co2_1}).

\begin{equation}
    \overline{A_\varepsilon}=\left[ \Delta p \cdot (M_\varepsilon \cdot 1_q)^T \cdot H_\varepsilon^{-1} \right] + A
    \label{eqn:co2}
\end{equation}

with

\begin{equation}
    H_\varepsilon = ( M_\varepsilon M_\varepsilon^T + I_n)
    \label{eqn:co2_1}
\end{equation}


Formula (\ref{eqn:co3}) with Formula (\ref{eqn:co4_1}) shows the Composition Operator without memory and with control \(\varepsilon\) and blending \(\delta\).
\begin{equation}
    \overline{A}^-=\left[ f_1^{\delta} \cdot \Delta p \cdot (M_\varepsilon \cdot 1_q)^T \cdot H_{\varepsilon \delta}^{-1} \right] + A
    \label{eqn:co3}
\end{equation}

Formula (\ref{eqn:co4}) with Formula (\ref{eqn:co4_1}) shows the Composition Operator with memory, control \(\varepsilon\) and blending \(\delta\).
\begin{equation}
    \overline{A}^+=\left[ f_1^{\delta} \cdot \Delta p \cdot (M_\varepsilon \cdot 1_q)^T + f_2^{\delta} \cdot P_q Q^T) + A (I_N + f_1^{\delta} \cdot M_\varepsilon M_\varepsilon^T \right] H_{\varepsilon \delta}^{-1}  
    \label{eqn:co4}
\end{equation}

with

\begin{equation}
    H_{\varepsilon \delta} = (f_1^\delta \cdot  M_\varepsilon M_\varepsilon^T + f_2^\delta \cdot QQ^T + I_n)
    \label{eqn:co4_1}
\end{equation}



This shift vector \(\Delta p\) is influenced by three different components. 
\begin{enumerate}
    \item distance \(d_{ij}\) between a chosen centroid \(C_i\) and its nearest centroid \(C_j\)
    \item distance \(d_{iC}\) between a chosen centroid \(C_i\) and the central centroid \(C_C\)
    \item randomly generated noise factor, following a normal distribution between -1 and 1
\end{enumerate}


The shift vector \(\Delta p\) is calculated by the following Formula (\ref{eqn:approach_1}).

\begin{equation}
    \Delta p = \underbrace{C_i - C_j}_{d_{ij}} + \frac{100}{100+n} \cdot \underbrace{C_i - C_C}_{d_{iC}} + \frac{100}{100+n} \cdot noise
    \label{eqn:approach_1}
\end{equation}
 as shown in Formula (\ref{eqn:approach_1_1}).

\begin{equation}
    \Delta p = \underbrace{\frac{\|d_{iC}\|^2_2}{(\|d_{ij}\|^2_2 + \|d_{iC}\|^2_2)} \cdot d_{ij}}_{v_{ij}} + \frac{100}{100+n} \cdot \underbrace{\frac{\|d_{ij}\|^2_2}{ (\|d_{ij}\|^2_2 + \|d_{iC}\|^2_2)} \cdot d_{iC}}_{v_{ij}} + \frac{100}{100+n} \cdot noise
    \label{eqn:approach_1_1}
\end{equation}


\begin{equation}
    DSC = \frac{|\{ p_{q,i} : \ \| p_{q,i} - C_q \| \leq \| p_{q,i} - C_k \| \ \ \forall \ k \in  |\{M\}| \}|}{N}
    \label{eqn:approach_4}
\end{equation}



\begin{equation}
    CD = \sum_{q}\sum_{i} \|C_q - p_q,i\|
    \label{eqn:approach_5}
\end{equation}
