\chapter{Anwendung des Harvardkonzepts}
Harvard Prinzip 1: Trennung von Sach- und Beziehungsebene
Das Ziel ist, Spannungen zu reduzieren, Rechtfertigungen zu vermeiden, und ein konstruktives Klima zu schaffen.
Beispiele von sachgerechten Aussagen von Frau Schneider / Frau Klar:
“Uns ist wichtig zu betonen, dass wir nicht einzelne Personen kritisieren, sondern vielmehr die Arbeitsprozesse, die sich seit der Umstrukturierung als schwierig erwiesen haben”
“Wir schätzen die Ziele der Umstrukturierung und möchten gemeinsam überlegen, wie diese in unserer täglichen Arbeit besser umgesetzt werden können.”
Diese Aussagen entlasten die Beziehungsebene, vermeiden Vorwürfe und nehmen den Vorgesetzten den Anlass zur Verteidigung. Sie signalisieren Kooperationsbereitschaft und Respekt.

Harvard Prinzip 2: Fokus auf Interessen statt Positionen
Das Ziel ist, Weg von Beschwerden, hin zu verständlichen Bedürfnisse.
Beispiele von sachgerechten Aussagen von Frau Schneider / Frau Klar:
“Unser Hauptanliegen ist ein reibungsloser Informationsfluss, damit wir unsere Aufgaben effizient und fehlerfrei ausführen können.”
“Für uns ist es wichtig, klare Zuständigkeiten zu haben, damit Doppelarbeit und unnötige Rückfragen vermieden werden können.”
Mit dieser Aussagen erhalten Herr Köhler und Herr Mahler eine klare Orientierung, worum es den Asssistentinnen wirklich geht, ohne sich angegriffen zu fühlen. Interessen sind leichter verhandelbar als Positionen.

Harvard Prinzip 3: Entwicklung von Optionen statt vorschneller Bewertungen
Das Ziel ist, gemeinsame Lösungsfindung zu ermöglichen.
Beispiele von sachgerechten Aussagen von Frau Schneider / Frau Klar:
“Vielleicht können wir gemeinsam überlegen, wie wir die Schnittstelle zum Service Center verbessern können.”
“Eine Möglichkeit wäre, bestimmte Aufgaben weiterhin direkt selbst zu erledigen und andere klar zu standardisieren. Eine andere Möglichkeit wäre, eine Testphase mit festen Ansprechpartnern im Service Center durchzuführen.”
Diese Aussagen laden die Vorgesetzten aktiv zur Mitgestaltung ein. Besonders Herr Mahler wird durch den kooperativen Ton angesproche, während Herr Köhler durch die Struktur der Optionen abgeholt wird.

Harvard Prinzip 4: Nutzung objektiver Kriterien 
Das Ziel ist, subjektive Meinungen durch sachliche Maßstäbe ergänzen.
Beispiele von sachgerechten Aussagen von Frau Schneider / Frau Klar:
“Wir könnten uns an objektiven Kriterien orientieren, etwa an Bearbeitungszeiten, Anzahl der Rückfragen oder Fehlerquoten.”
“Vielleicht lassen sich interne Ergebnisse aus anderen Abteilungen heranziehen, um eine Vergleichbasis zu schaffen.”
Objektive kriterien helfen insbesodere Herrn Köhler, da sie seine sachlich-rationale Orientierung ansprechen. Gleichzeitig reduzieren sie emotionale Spannungen.