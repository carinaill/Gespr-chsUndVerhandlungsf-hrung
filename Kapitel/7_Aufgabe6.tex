\chapter{Analyse einer Verhandlungssituation}
\section{Aufgabe 6a (aqui puedes poner el titulo cuando lo cambies)}
In der vorliegenden Gesprächssituation findet keine Verhandlung im eigentlichen Sinne statt. Eine Verhandlung erfordert das gemeinsame Ziel, eine verbindliche Einigung zu erzielen, den Austausch von konkreten Angeboten und Gegenangeboten sowie eine systematische Klärung der Interessen der Beteiligten. Im Fallbeispiel dominieren jedoch überwiegen argumentativ geführte Wortwechsel, in denen insbesondere die beiden Vorgesetzten versuchen, ihre Sichtweise zu verteidigen, anstatt lösungsorientiert auf die Anliegen der Teamassistentinnen einzugehen.
Im Rahmen des Seminars wurden drei Schritte der inhaltlichen Verhandlungsvorbereitung vermittelt: Schritte 1: Identifizieren beteiligter Personen/-kreise Feststellung; Schritte 2: Klärung der Interessen; Schritte 3: Suche nach tieferliegenden Motiven und Bedürfnissen nach Maslow. Bereits in Schritt 1 zeigt sich, dass grundlegende Voraussetzungen für eine erfolgreiche Verhandlung nicht erfüllt sind. Zwar sind die relevanten Personen am Tisch vertreten, jedoch werden ihre Interessen nicht angemessen berücksichtigt oder sichtbar gemacht. Stattdessen wird der Gesprächsfluss durch zahlreiche Unterbrechungen gestört, sowohl zwischen den Parteien als auch innerhalb derselben Seite. So unterbricht Frau Schneider und Herr Mahler gegenseitig. Diese Kommunikationsmuster verhindern, dass Interessen klar benannt und gehört werden. 
Aktives Zuhören, ein zentraler Bestandteil konstruktiver Verhandlungsführung, ist im gesamten Gespräch kaum erkennbar. Gute Verhandlungsführung setzt voraus, dass die Parteien einander aufmekrsam zuhören, um Perspektiven, Bedürfnisse und Beweggründe zu verstehen. Im Fallbeispiel passiert das Gegenteil, Aussagen der Teamassistentinnen werden umgedeutet oder ignoriert. Ein Beispiel hierfür ist die Reaktion von Herrn Köhler auf Frau Klars Hinweis zum Informationsverlust. Anstatt das Problem strukturell zu hinterfragen, formuliert er das Thema in ein Kompetenzargument um. Zum Beispiel: „Also, Frau Klar, Sie sitzen doch an zentraler Stelle und koordinieren die Arbeit der Projektmanager auch mit externen Projektpartnern, dann ist es doch auch kein Problem, die Arbeit intern zu koordinieren…”. Damit wird weder das Interesse der Teamassistentinnen aufgegriffen, noch wird ein Lösungsansatz entwickelt.
Auch der zweite und dritte Schritte der Verhandlungsvorbereitung, die Klärung der Interessen sowie die Analyse tieferliegender Bedürfnisse, finden im Gespräch nicht statt. Weder formulieren die Assistentinnen ihre Kerninteressen strukturiert, noch zeigt die Gegenseite Bereitschaft, sich mit deren Anliegen vertieft auseinanderzusetzen. Statt eines Austauschs von Interessen werden allgemeine Rechtfertigungen für die Umstrukturierung gegeben.
Zwar weist Kreggenfeld darauf hin, dass argumentative Sequenzen ein konstitutiver Bestandteil vieler Verhandlungen sein können. In echten Verhandlungen dienen Argumente jedoch dazu, Positionen zu begründen, Forderungen zu legitimieren und den Anliegen der Gegenseite stichhaltig und lösungsorientier zu begegnen. Genau dies fehlt im vorliegenden Fall. Die Aussagen der Teamassistentinnen werden nicht validiert, sondern abgeschwächt. Das zeigt, dass zwar argumentiert wird, jedoch nicht im Sinne eines verhandlungsorientierten Austauschs, sondern vielmehr im Sinne eines Verteidigens der eigenen Position.
Da somit sowohl die strukturellen Merkmale einer Verhandlung (Ziel einer Einigung, Austausch von Angeboten, Interessenklärung) als auch die kommunikativen Voraussetzungen (aktives Zuhören, gegenseitige Anerkennung von Interessen) fehlen, handelt es sich nicht um eine echte Verhandlung, sondern um eine argumentativ geführte Diskussion, der die zentralen Definitionskriterien nicht erfüllt.

\section{Aufgabe 6b}
1. Schritt: Feststellung der relevanten Parteien
Im Fallbeispiel treten die Teamassistentinnen als Verhandelnde auf, während die Vorgesetzten die Gegenseite bilden. Gegenstand der potenziellen Verhandlung ist die Aufgabenverteilung nach der Umstrukturierung sowie die Zusammenarbeit mit dem Service Center, insbesondere im Hinblick auf Informationsfluss, Arbeitsbelastung und Verantwortlichkeit.\\

2. Schritt: Klärung der Interessen
Es werden die persönlichen und geschäftlichen Interessen beider Seiten analysiert sowie die Interessen weiterer Beteiligter identifiziert.\\

3. Schritt: Bedürfnisse und Motive
Es werden die tieferliegenden Motive betrachtet, die hinter den Interessen stehen. Hier können Modelle wie Maslows Bedürfnispyramide herangezogen werden.\\


\section{Aufgabe 6c}
Im Gesprächsverlauf lassen sich mehrere Stellen identifizieren, an denen erkennbar wird, dass das Verhalten, die Formulierungen und die Schwerpunktsetzung der Teamassistentinnen nicht zu einem zielführenden Verlauf beitragen. Im Folgenden wird gezeigt, dass sowohl kommunikative Grundprinzipien als auch zentrale Elemente erfolgreicher Verhandlungsführung verletzt werden.\\

Punkt 1: Häufige Unterbrechungen und fehlendes aktives Zuhören
Der Gesprächsfluss wird mehrfach durch Unterbrechungen gestört, sowohl zwischen den Parteien als auch innerhalb derselben Seite. Besonders auffällig ist, dass Frau Schneider ihre Kollegin Frau Klar unterbricht, obwohl beide die gleiche Position haben. Auch zwischen Frau Schneider und Herrn Mahler kommt es zu Unterbrechungen.
Dieses Verhalten widerspricht Watzlawicks Axiom 2, demzufolge jede Kommunikation einen Inhalts- und einen Beziehungsaspekt besitzt. Durch Unterbrechungen wird der Beziehungsteil negativ beeinflusst, die Teamassistentinnen erscheinen ungeduldig und wenig kooperativ, während die Vorgesetzten dominant wirken.
Hier wird insbesondere Prinzip 1 von der Harvard Konzept (Menschen und Probleme getrennt behandeln) verletzt. Die Assistentinnen reagieren auf Aussagen der Vorgesetzten nicht sachlich, sonder impulsiv und auf der Beziehungsebene, wodurch die Gegenseite in eine Verteidigungshaltung gerät.
Unterbrechungen verhindern, dass Argumente klar formuliert werden, und reduzieren die Glaubwürdigkeit der Assistentinnen.\\

Punkt 2: Emotional gefärbte Formulierungen und fehlende Struktur in der Darstellung
Frau Schneider formuliert emotional aufgeladene Aussagen, wie zum Beispiel:“Das kann ich nicht als Kosteneinsparung akzeptieren.“
Auch Frau Klar formuliert unstrukturierte Problembeschreibungen:“bevor ich es lange erklären muss…. Da kommen nur wieder Rückfragen an mich.”
Statt klarer Interessen oder Forderungen werden überwiegend Beschwerden geäußert.
Die stark emotionale Kommunikation verstärkt den Beziehungsaskpekt (Axiom 2). Die Vorgesetzten reagieren darauf mit Rechtfertigung und Abwehr, wodurch eine Eskalationsspirale im Sinne von Axiom 3 entsteht.
Im Bezug zum Harvard Konzept, beim Prinzip 2 (Konzentrieren Sie sich nicht auf Positionen, sondern auf dahinterliegende Interessen), kommunizieren die Assistentinnen jedoch Positionen, wie zum Beispiel: “Wir haben mehr Arbeit” oder “Es funktioniert nicht”, statt deren Interessen präzise zu benennen. Zum Beispiel, klare Verantwortlichkeiten oder effizienter Informationsfluss.\\

Punkt 3: Fokussierung auf Symptome statt strategische Interessen
Die Assistentinnen sprechen vor allem über Probleme wie Informationsverlust, Arbeitsüberlastung, Rückfragen des Service Centers oder unterschiedliche Arbeitsweisen der Projektleiter. Aber sie benennen keinen klaren Interessen oder Ziele. Hier wäre es hilfreich zu fragen: “Was soll sich konkret ändern?”, “Welche Struktur wäre hilfreich?”, “Welche Prozesse sollten angepasst werden?”. Dadurch entsteht kein gestaltbarer Verhandlungsgegenstand.
Die Vorgesetzten greifen diese Darstellungen auf und deutem sie zum Teil um. Ein Beispiel ist Herr Köhler, der Frau Klars strukturellen Hinweis fälschlicherweise als Kompetenzproblem interpretiert (“…es sollte kein Problem sein, intern zu koordinieren”).
Beim Harvard Konzept werden Prinzip 2 und auch Prinzip 3 (Entwickeln Sie Optionen, bevor Sie entscheiden) verletzt. Weder werden Interessen klar geäußert, noch entstehen Optionen.
Solange die Assistentinnen keine Interesse formulieren, können die Vorgesetzten sie leicht relativieren. Die Diskussion bleibt oberflächlich und führt zu keiner Lösungsentwicklung.\\

Damit entsteht kein echter Verhandlungsprozess im Sinne der Theorie, sondern eine argumentativ geprägte Auseinandersetzung ohne eindeutige Zielführung.

\section{Aufgabe 6d}
Die Verbesserungsmöglichkeiten der Teamassistentinnen: \\
Verbesserung der Vorbereitung:\\
Die Teamassistentinnen hätten sich strukturierter auf das Gespräch vorbereiten müssen. Troczynski betont, dass eine erfolgreiche Verhandlungsvorbereitung voraussetzt, sich vorab intensiv mit den eigenen Interessen, den möglichen Interessen der Gegenseite, sowie mit deren erwartbaren Reaktionen auseinandersetzen. Im Fallbeispiel die Assistentinnen benennen Probleme, diese jedoch nicht in klar formulierte Interesse oder Ziele übersetzen.\\
Statt überwiegend operative Schwierigkeiten zu schildern, hätten sie beispielweise definieren können, wie zum Beispiel, welche konkreten Veränderungen sie anstreben, welce Prozesse angepasst werden sollen, welche Entlastung sie benötigen und warum.
Daüber hinaus wäre es sinnvoll gewesen, im Vorfeld eine gemeinsame Argumentationslinie abzustimmen. Die gegenseitigen Unterbrechungen zwischen Frau Schneider und Frau Klar deuten auf eine unzureichende interene Abstimmung hin, die Troczynski ausdürcklich als Risiko für eine erfolgreiche Verhandlungsführung beschreibt.\\
Verbesserung der Gesprächs - und Verhandlungsführung:\\
In der Gesprächsführung hätten die Teamassitentinnen stärker auf Sachlichkeit, Struktur und aktives Zuhören achten müssen. Emotionale und Formulierungen wie “Das kann ich nicht akzeptieren” vermischen Sach- und Beziehungsebene und führen dazu, dass die Gegenseite in eine Verteidigungshaltung gerät.
Darüber hinaus wäre es sinnvoll gewesen, ruhig zuzuhören, die eigenen Anliegen klar zu strukturieren und bewusst lösungsorientierte Beiträge zu formulieren.\\

Die Verbesserungsmöglichkeiten der Vorgesetzten:\\
Verbesserung der Vorbereitung:\\
Die Vorgesetzten hätten sich gründlicher auf das Gespräch vorbereiten müssen. Troczynski betont, dass Führungskräfte vor Gesprächen mit Mitarbeitende deren Perspektive ernsthaft analysieren sollten. Im Fallbeispiel zeigt jedoch, dass Herr Mahler und Herr Köhler die Kritik der Assistentinnen offenbar nicht systematisch im Voraus bewertet haben.
Anstatt in erster Linie die Erfolge der Umstrukturierung hervorzuheben, hätten sie sich vorbereitende Fragen stellen können, wie zum Beispiel, welche konkreten Probleme treten in der täglichen Arbeit auf, wo könnte ein echter Anpassungsbedarf bestehen, welche Punkte sind verhandelbar. Eine solche Vorbereitung hätte eine offenere Haltung in den Diskussionen ermöglicht.\\
Verbesserung der Gesprächs- und Verhandlungsführung:\\
In der Gesprächsführung neigen Vorgesetzte dazu, defensiv zu reagieren. Sie antworten auf Kritik mit Rechtfertigungen oder Umdeutungen, anstatt zunächst die vorgebrachten Bedenken anzuerkennen. Troczynski betont jedoch, dass aktives Zuhören, Perspektivwechsel und die Validierung der anderen Seite Schlüsselelemente einer konstruktiver Gesprächsführung sind.
Zielführender wäre es gewesen, das Gespräch zu moderieren, Themen zu strukturieren und gemeinsam Optionen zu entwickeln. Dies entspricht dem 3. Prinzip des Harvard-Konzepts, das fordert, zunächst möglichst viele Lösungsoptionen zu entwickeln, bevor Entscheidungen getroffen werden.\\

Zusammenfassend lässt sich feststellen, dass das Gespräch scheitert, weil beide Seiten grundlegende Prinzipien erfolgreicher Verhandlungsführung vernachlässigen:\\
fehlende strukturierte Vorbereitung(Troczynski)\\
mangelnde Trennung von Sach- und Beziehungsebene(Harvard Prinzip 1)\\
unzureichende Interessenklärung(Harvard Prinzip 2)\\
fehlende Entwicklung von Lösungsoptionen(Harvard Prinzip 3)\\
Defizite in der Kommunikation und im Zuhören(Watzlawick)\\
Eine frühzeitige, reflektierte Vorbereitung sowie eine sachliche, strukturierte und empathische Gesprächsführung hätten auf beiden Seite dazu beitragen können, das Gespräch deutlich zielführender zu gestalten.

\section{Aufgabe 6e}
Im Folgenden werden die beiden Vorgesetzten Herr Köhler und Herr Mahler auf Grundlage ihres Verhaltens im Gespräch tedenziell Persönlichkeitstypen nach dem Riemann-Thomann-Kreuz zugeordnet. \\

Herr Köhler:\\
Tedenzieller Persönlichkeitstyp: Distanz -/Dauer orientierter Typ\\
Herr Köhler tritt im Gespräch sachlich-nüchtern, rational und kontrollierend auf. Er argumentiert stark aus einer strukturellen und funktionalen Perspektive heraus und zeigt wenig Bereitschaft, emotionale oder subjektive Aspekte aufzugreifen. Auffällig ist, dass er Probleme häufig abstrahiert und verallgemeinert, etwa wenn er die geschilderten Schwierigkeiten auf die Art der Informationsweitergabe reduziert oder implizit als Kompetenzfrage der Teamassistentinnen umdeutet.\\
Dieses Verhalten entspricht Merkmalen des Distanz-Typs im Rieman-Thomann-Modell:\\
Bedürfnis nach Sachlichkeit und Klarheit\\
geringe Affinität zu emotionalen Argumenten\\
Fokus auf Systeme, Prozesse und Logik\\
Tendenz, Nähe und persönliche Betroffenheit zu vermeiden\\
Wie sollte man Herrn Köhler ansprechen? \\
sachlich, ruhig und strukturiert\\
klare Argumente, Fakten und Beispiele\\
kleine emotionalen Vorwürfe \\
Fokus auf Prozesse, Effizienz und Systemlogik\\
Probleme als Strukturthemen, nicht als persönliche Belastung darstellen\\

Herr Mahler: \\
Tedenzieller Persönlichkeitstyp: Nähe- / Dauer-orientierter Typ\\
Herr Mahler zeigt im Gespräch eine grundsätzlich zugewandtere und verständnisvollere Haltung. Er signalisiert zuminderst verbal, dass er die Probleme kennt und versucht, Spannungen zu relativieren. Gleichzeitig vermeidet er klare Entscheidungen oder konkrete Lösungsvorschläge und verweist wiederholt auf Entwicklungsprozesse, Gewöhnung und Teambuilding.\\
Dieses Verhalten entspricht Merkmalen des Nähe-Typs im Rieman-Thomann-Modell:\\
Wunsch nach Harmonie und Ausgleich\\
Vermeidung offener Konfrontation\\
Anerkennung von Gefühlen, jedoch ohne Konsequenzen\\
Bedürfnis nach Zustimmung und sozialer Stabilität\\
Wie sollte man Herrn Mahler ansprechen?\\
wertschätzend und respektvoll\\
Beziehungsebene aktiv berücksichtigen\\
Zustimmung und Verständnis spiegeln\\
Anliegen klar formulieren, aber ohne Konfrontation\\
gemeinsame Ziele betonen\\

Die Teamassistentinnen sollten sich vor dem Gespräch bewusst machen, welcher Typ ihnen gegenübersitzt, welche Kommunikationsform dieser bevorzugt, wie sie ihre Inhalte entsprechen übersetzen können. Das Riemann-Thomann-Modell verdeutlicht, dass gleiche Inhalte je nach Persönlichkeitstyp unterschiedlich vermittelt werden müssen, um wirksam zu sein.\\

Die Analyse nach dem Modell zeigt, dass das Gespräch auch deshalb scheitert, weil die Assistentinnen ihre Kommunikationsweise nicht ausreichend an die Persönlichkeitstypen der Vorgesetzten angepasst haben. Während Herr Köhler sachlich-strukturell angesprochen werden müsste, benötigt Herr Mahler eine stärker beziehungsorientierte Ansprache. Eine bewusste Vorbereitung unter Berücksichtigung dieser Unterschiede hätte dazu beigetragen können, das Gespräch könstruktiver und zielführender zu gestalten\\.