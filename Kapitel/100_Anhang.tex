\chapter{Pseudocode for Heuristics}
\label{appendix:pseudocode_heuristics}

\begin{table}[H]
\centering
    \fbox{%
    \begin{tabular}{ c | p{5cm} | c }
    \textbf{3. Schritt:} \\ \textbf{Bedürfnisse und Motive} & \textbf{Tieferliegende} \\ \textbf{Interessen}   & \textbf{Relative Bedeutung} \\
    \hline 
    \hline
    \textbf{Meine Interessen:} &  & \\
    \hline
    Anerkennung & Wertschätzung, Status & 80\\
    \hline
    klare Kommunikation & Struktur, Sicherheit, weniger Stress & 100\\
    \hline
    weniger Fehler & Professionalität, Sicherheit & 100\\
    \hline
    effiziente Prozesse & Kompetenz erleben Wirksamkeit, Planbarkeit & 95\\
    \hline
    \hline
    \textbf{Interessen der Gegenseite:}  & & \\
    \hline
    Anerkennung & ökonomische Sicherheit & 100\\
    \hline
    Umstrukturierung verteidigen & Geld/Zeit sparen, Autorität & 100\\
    \hline
    Einheitliche Prozesse & Kontrolle, Ordnung & 95\\
    \hline
    positive Außenwirkung & Reputation & 90\\
    \end{tabular}
    }%
    \caption{Bedürfnisse und Motive}
    \label{tab:standard_comparison_var1}
\end{table}
