\chapter{Pseudocode for Heuristics}
\label{appendix:pseudocode_heuristics}
    
\begin{table}[h]
    \centering
    \renewcommand{\arraystretch}{1.3}
        
    \begin{tabularx}{\textwidth}{|>{\raggedright\arraybackslash}X|
                                  >{\raggedright\arraybackslash}X|}
    \hline
    \multirow{3}{=}{\textbf{1. Schritt:}\\
    Feststellung der relevanten Parteien}
    &
    \textbf{Verhandelnder:} \textit{Teamassistentinnen} \\ \cline{2-2}
        
    &
    \textbf{Gegenseite:} \textit{Vorgesetzte} \\ \cline{2-2}
        
    &
    \textbf{Gegenstand:} Aufgabenverteilung \\
    \hline
    \textbf{Personen auf „meiner Seite“, die an dem Ergebnis interessiert sein könnten:}
    &
    \textbf{Personen auf der „Gegenseite“, die an dem Ergebnis interessiert sein könnten:} \\
    \hline
    Interessengruppen? Teamassistenz-Kolleginnen
    &
    Interessengruppen? Geschäftsführung / Service Center \\
    \hline
    Freunde? nicht relevant
    &
    Freunde? nicht relevant \\
    \hline
    Familie? indirekt betroffen durch Stress
    &
    Familie? nicht relevant \\
    \hline
    Chef? Projektleiter
    &
    Chef? höhere Führungsebene, Unternehmensleitung \\
    \hline
    Sonstige? Kunden
    &
    Sonstige? andere Abteilungen, die mit der Umstrukturierung zufrieden sind \\
    \hline
    \dots
    &
    \dots \\
    \hline
    \end{tabularx}
    \end{table}
        
    

\begin{table}[H]
\centering
    \fbox{%
    \begin{tabular}{ c | p{5cm} | c }
    \textbf{3. Schritt:} \\ \textbf{Bedürfnisse und Motive} & \textbf{Tieferliegende} \\ \textbf{Interessen}   & \textbf{Relative Bedeutung} \\
    \hline 
    \hline
    \textbf{Meine Interessen:} &  & \\
    \hline
    Anerkennung & Wertschätzung, Status & 80\\
    \hline
    klare Kommunikation & Struktur, Sicherheit, weniger Stress & 100\\
    \hline
    weniger Fehler & Professionalität, Sicherheit & 100\\
    \hline
    effiziente Prozesse & Kompetenz erleben Wirksamkeit, Planbarkeit & 95\\
    \hline
    \hline
    \textbf{Interessen der Gegenseite:}  & & \\
    \hline
    Anerkennung & ökonomische Sicherheit & 100\\
    \hline
    Umstrukturierung verteidigen & Geld/Zeit sparen, Autorität & 100\\
    \hline
    Einheitliche Prozesse & Kontrolle, Ordnung & 95\\
    \hline
    positive Außenwirkung & Reputation & 90\\
    \end{tabular}
    }%
    \caption{Bedürfnisse und Motive}
    \label{tab:standard_comparison_var1}
\end{table}
