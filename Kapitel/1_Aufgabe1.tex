% ----------------------------------------------------------
% ----------------------------------------------------------
\let\clearpage\relax\vspace{3\baselineskip}
\chapter{Verhandlungen mit einem potentiellen Lieferanten}

\section{a: Strategische Wahl des Gesprächsortes}
\label{sec:gesprächsortswahl}
Bei Verhandlungen mit einem potenziellen neuen Lieferanten sollte der Ort des Gesprächs bewusst gewählt werden, da er den Verhandlungsverlauf beeinflussen kann. 
Der Verhandlungsort wirkt nicht nur organisatorisch, sondern auch auf das Auftreten und das Sicherheitsgefühl der Beteiligten.\footnote{Vgl. \cite{vanderwijst2020}, Introduction.}\\
Ein zentraler Aspekt ist dabei das Macht- und Kontrollverhältnis. 
Treffen am eigenen Unternehmensstandort können einen sogenannten „Home-Field Advantage“ erzeugen,
da sich die eigene Seite in vertrauter Umgebung meist sicherer und souveräner fühlt.\footnote{Vgl. \cite{vanderwijst2020}, Abschnitt „Home-Field Advantage“.}
Dies kann helfen, eine starke Verhandlungsposition zu signalisieren. Gleichzeitig besteht jedoch die Gefahr, 
dass der Lieferant dies als Machtdemonstration wahrnimmt, was sich negativ auf die Beziehung auswirken kann.\\
Aus diesem Grund kann insbesondere bei Erstverhandlungen ein neutraler Ort sinnvoll sein. 
Neutrale Umgebungen werden häufig als fairer wahrgenommen, was Vertrauen fördern und den Gesprächseinstieg erleichtern kann.\footnote{Vgl.\cite{vanderwijst2020}: in \cite{filzmoser2020}, S. 1-24.}
Zudem können praktische Aspekte wie gute Erreichbarkeit, angemessene Meetingräume und vorhandene Präsentationstechnik die Professionalität des Gesprächs unterstützen.\\
Darüber hinaus spielt auch die Atmosphäre des Verhandlungsortes eine Rolle. 
Angenehme und ruhige Umgebungen können Stress reduzieren und eine kooperative Gesprächsatmosphäre fördern, 
was vor allem bei einer angestrebten langfristigen Lieferantenbeziehung von Vorteil ist.\footnote{Vgl. \cite{vanderwijst2020}, Negotiation and the Role of Location.} \\
Vor diesem Hintergrund erscheint besonders bei einem neuen Lieferanten ein neutraler, professioneller Ort als geeignete Lösung

\section{b: Informierung zur Terminvorbereitung}
\label{sec:terminvorbereitung}
%Kreggenfeld:
Zur Vorbereitung auf den ersten Verhandlungstermin analysiere ich zunächst den Verhandlungsgegenstand, also die Rahmenbedingungen der Lieferungen.
Ich recherchiere Branchenpreise, Lieferzeiten sowie mögliche alternative Lieferanten, um meine Verhandlungsspielräume und Prioritäten 
(Muss-, Soll-, Kann-Ziele) klar zu definieren.\footnote{Vgl. \cite{kreggenfeld2021}, S.210} 
Dabei kläre ich außerdem mit der zugehörigen Abteilung meines Unternehmens, wie viel Verhandlungsspielraum ich tatsächlich habe, 
also wie weit ich dem Lieferanten entgegenkommen darf und auf welche Aspekte auf keinen Fall verzichtet werden kann.
Ich untersuche auch die Ziele und Interessen meines Verhandlungspartners von seiner Webseite, 
mögliche Verhaltensmuster sowie Hinweise aus sozialen Netzwerken wie LinkedIn oder Kununu. 
Diese kategorisiere ich, wodurch ich Gemeinsamkeiten und Unterschiede zwischen unseren Interessen herausarbeiten
und mögliche Lösungsansätze ableiten kann. \\
Im Rahmen eines umfangreichen Profilings informiere ich mich über aktuelle Zahlen des Lieferanten (Liefervolumen, Umsatz, Gewinn), 
dem Markt (Geschäftsfelder, Marktstellung, Zielgruppen), aktuelle Situation des Lieferanten und die Chancen und Risiken der Zusammenarbeit.\footnote{Vgl. \cite{troczynski2023}, 3.6.1, S. 83-86}
Falls ich Informationen über meinen direkten Verhandlungsgegenüber ausmachen kann, 
berücksichtige ich subtil Techniken wie die Beobachtung von Körpersprache oder Mikroexpressionen, 
um den Typ einschätzen zu können.\footnote{Vgl. \cite{troczynski2023}, Kap. 3.8.2., S. 111-127.}. \\
Aus der Recherche entscheide ich, ob ich ein ergebnis- oder beziehungsorientiertes Ziel verfolge 
und ziehe Inforamtionen über passende Strategien (z.B Win-Win) heran. \\
Zudem wähle ich einen geeigneten Verhandlungsort, mit Hilfe der Ergebnisse aus der Antwort zu 1a.
Außerdem recherchiere ich wie lang ein Verhandlungsgespräch in meiner Branche durchschnittlich dauert und wann die beste Uhrzeit dafür ist.\\
