% ----------------------------------------------------------
{\let\clearpage\relax\vspace{3\baselineskip}
\chapter{Verhandlungen mit einem potentiellen Lieferanten}}

\section{a: Strategische Wahl des Gesprächsortes}
\label{sec:gesprächsortswahl}
Bei Verhandlungen mit einem potenziellen neuen Lieferanten sollte der Ort des Gesprächs bewusst gewählt werden, da er den Verhandlungsverlauf beeinflussen kann. 
Der Verhandlungsort wirkt nicht nur organisatorisch, sondern auch auf das Auftreten und das Sicherheitsgefühl der Beteiligten.\footnote{Vgl. \cite{vanderwijst2020}, Introduction.}\\
Ein zentraler Aspekt ist dabei das Macht- und Kontrollverhältnis. 
Treffen am eigenen Unternehmensstandort können einen sogenannten „Home-Field Advantage“ erzeugen,
da sich die eigene Seite in vertrauter Umgebung meist sicherer und souveräner fühlt.\footnote{Vgl. \cite{vanderwijst2020}, Abschnitt „Home-Field Advantage“.}
Dies kann helfen, eine starke Verhandlungsposition zu signalisieren. Gleichzeitig besteht jedoch die Gefahr, 
dass der Lieferant dies als Machtdemonstration wahrnimmt, was sich negativ auf die Beziehung auswirken kann.\\
Aus diesem Grund kann insbesondere bei Erstverhandlungen ein neutraler Ort sinnvoll sein. 
Neutrale Umgebungen werden häufig als fairer wahrgenommen, was Vertrauen fördern und den Gesprächseinstieg erleichtern kann.\footnote{Vgl.\cite{vanderwijst2020}: in \cite{filzmoser2020}, S. 1-24.}
Zudem können praktische Aspekte wie gute Erreichbarkeit, angemessene Meetingräume und vorhandene Präsentationstechnik die Professionalität des Gesprächs unterstützen.\\
Darüber hinaus spielt auch die Atmosphäre des Verhandlungsortes eine Rolle. 
Angenehme und ruhige Umgebungen können Stress reduzieren und eine kooperative Gesprächsatmosphäre fördern, 
was vor allem bei einer angestrebten langfristigen Lieferantenbeziehung von Vorteil ist.\footnote{Vgl. \cite{vanderwijst2020}, Negotiation and the Role of Location.} \\
Vor diesem Hintergrund erscheint besonders bei einem neuen Lieferanten ein neutraler, professioneller Ort als geeignete Lösung.

\section{b: Informierung zur Terminvorbereitung}
\label{sec:terminvorbereitung}
%Kreggenfeld:
Zur Vorbereitung auf den ersten Verhandlungstermin erfolgt zunächst eine Analyse des Verhandlungsgegenstands, insbesondere der Rahmenbedingungen der Lieferungen.
Dazu werden Branchenpreise, Lieferzeiten sowie mögliche alternative Lieferanten recherchiert, um die Verhandlungsspielräume und Prioritäten 
(Muss-, Soll-, Kann-Ziele) klar zu definieren.\footnote{Vgl. \cite{kreggenfeld2021}, S.210} 
Darüber hinaus mit der zuständigen Fachabteilung des Unternehmens der Verhandlungsspielraum abgestimmt,
also in welchem Umfang dem Lieferanten entgegengekommen werden darf und auf welche Aspekte keinesfalls verzichtet werden kann.
Dabei kläre ich außerdem mit der zugehörigen Abteilung meines Unternehmens, wie viel Verhandlungsspielraum ich tatsächlich habe, 
also wie weit ich dem Lieferanten entgegenkommen darf und auf welche Aspekte auf keinen Fall verzichtet werden kann.
Weiterhin werden die Ziele und Interessen des Verhandlungspartners anhand öffentlich zugänglicher Quellen, 
wie der Unternehmenswebseite sowie sozialer Netzwerke (z. B. LinkedIn oder Kununu), untersucht.  
Hinweise auf mögliche Verhaltensmuster werden systematisch kategorisiert, um Gemeinsamkeiten und Unterschiede zwischen den Interessen beider Parteien herauszuarbeiten 
und potenzielle Lösungsansätze abzuleiten. \\
Im Rahmen eines umfassenden Profilings werden zudem aktuelle Kennzahlen des Lieferanten (z. B. Liefervolumen, Umsatz, Gewinn), Informationen zum Markt (Geschäftsfelder, Marktstellung, Zielgruppen),
sowie die aktuelle Situation des Lieferanten einschließlich der Chancen und Risiken einer Zusammenarbeit untersucht.\footnote{Vgl. \cite{troczynski2023}, 3.6.1, S. 83-86}
Sofern Informationen über den direkten Verhandlungsgegenüber verfügbar sind, werden diese berücksichtigt.
Techniken wie die Beobachtung von Körpersprache oder Mikroexpressionen helfen, 
eine Einschätzung des Persönlichkeitstyps durchzuführen.\footnote{Vgl. \cite{troczynski2023}, Kap. 3.8.2., S. 111-127.}. \\
uf Basis der gewonnenen Erkenntnisse wird entschieden, ob ein ergebnis- oder beziehungsorientiertes Verhandlungsziel verfolgt wird.
In diesem Zusammenhang werden Informationen über geeignete Verhandlungsstrategien (z. B. Win-Win-Ansätze) herangezogen. \\
Zudem  erfolgt die Auswahl eines geeigneten Verhandlungsortes unter Berücksichtigung der Ergebnisse aus Antwort 1a. 
Ergänzend werden branchenübliche Verhandlungsdauern sowie geeignete Tageszeiten für das Verhandlungsgespräch recherchiert.er