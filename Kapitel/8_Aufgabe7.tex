\chapter{Verhandlungstypen erkennen}
Verhandlungstypen lassen sich insbesondere durch aufmerksame Beobachtung der Kommunikation und des Verhaltens am Verhandlungstisch identifizieren. Zentrale Hinweise sind:\\
der Umgang mit Positionen und Interessen\\
die Art der Argumentation (sachlich vs. emotional)\\
der Einsatz von Druck oder Beziehung
die Bereitschaft zu Zugeständnissen
der Grad an Offenheit oder Zurückhaltung
der Umgang mit Konflikten und Emotionen
Das Seminar unterscheidet unter anderem zwischen harten und weichen Verhandlungsstilen, wie sie auch im Harvard-Konzept beschrieben werden. Während harte Verhandlungstypen auf Durchsetzung, Kontrolle und das Beharren auf Positionen setzen, legen weiche Verhandlunstypen mehr Wert auf Beziehung, Harmonie und Kompromisse.

Am Verhandlungstisch kann man insbesondere darauf achten:
den Sprachstil(direkt, fordernd, vorsichtig, emotional)
die Reaktion auf Gegenargumente
das Verhalten bei Druck oder Widerstand
den Umgang mit Zeit, Pausen und Entscheidungen
sowie darauf, ob die Person eher Lösungen entwickeln oder Positionen verteidigen möchte
Mit dieser Beobachtungen kann man besser die Situation einzuschätzen und der Verhandlungstyp einzuordnen.

Das Erkennen von Verhandlungstypen hat direkten Einfluss auf eigenes Verhalten. Je nach Typ pass man seine Kommunikations- und Verhandlungsstrategie an:
Gegenüber harten, bestimmenden Verhandlungstypen ist eine sachliche, klare und strukturierte Argumentation sinnvoll. Emotionale Appelle oder voreilige Zugeständnisse sollten vermieden werden.
Gegenüber weichen, beziehungsorientierten Verhandlungstypen ist es hilfreich, auf die Beziehungsebene einzugehen, Vertrauen aufzubauen und kooperative Lösungsansätze zu betonen.
Bei zurückhaltenden oder defensiven Typen ist Geduld und aktives Zuhören erforderlich.
Bei offensiven Typen kann es notwendig sein, klare Grenzen zu setzen und gleichzeitig ruhig und souverän zu bleiben
Es ist nicht das Ziel, den eigenen Stil aufzugeben, sondern flexibel zu agieren und das eigene Verhalten bewusst auf den jeweiligen Verhandlungspartener abzustimmen. Das Seminar betont, dass sachbezogenes und prinzipiengeleitetes Verhandeln langfristig erfolgreicher ist als starres Festhalten an einem bestimmten Stil.