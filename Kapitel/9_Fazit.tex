{\let\clearpage\relax\vspace{3\baselineskip}
\chapter*{Fazit}}
\addcontentsline{toc}{chapter}{Fazit}
Die Hausarbeit zeigt, dass erfolgreiche Verhandlungen Vorbereitung, klare Ziele und aktives Zuhören erfordern. 
Durch das Harvard-Konzept wird deutlich, dass Lösungen vor allem durch Interessenklärung und kooperative Ansätze entstehen. 
Kommunikationsprobleme, Emotionen und unterschiedliche Persönlichkeitstypen können Konflikte verstärken. 
Modelle wie Watzlawick oder das Riemann-Thomann-Kreuz helfen, Gesprächsverläufe besser einzuschätzen und das eigene Verhalten 
anzupassen. Insgesamt ist Verhandeln lernbar und hängt von Reflexion, Flexibilität und Orientierung am Gegenüber ab.\\
Die Arbeit entstand in enger Zusammenarbeit und regelmäßiger Kommunikation.
Die Aufgaben wurden koordiniert aufgeteilt, sodass die Arbeitslast gleichmäßig verteilt war und
beide Autorinnen jeweils etwa die Hälfte übernommen haben. Texte wurden getrennt verfasst und anschließend gemeinsam überarbeitet, 
sodass beide Autorinnen gleichermaßen zu den Ergebnissen beigetragen haben.
