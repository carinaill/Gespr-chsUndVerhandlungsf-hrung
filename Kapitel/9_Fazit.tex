{\let\clearpage\relax\vspace{3\baselineskip}
\chapter*{Fazit}}
\addcontentsline{toc}{chapter}{Fazit}
Die Hausarbeit verdeutlicht, dass erfolgreiche Verhandlungen ein strukturiertes Vorgehen sowie ausgeprägte Kommunikation und analytische Fähigkeiten erfordern. Verhandlungen sind keine spontane Gespräche, sondern gezielte Prozesse, die eine bewusste Vorbereitung, klar Zielsetzungen und eine reflektierte Gesprächsführung voraussetzen. \\
Die Auseinenadersetzung mit Verhandlungsvorbereitung, aktivem Zuhören und sachgerechtem Verhandeln nach dem Harvard-Konzept zeigt, dass nachhaltige Lösungen vor allem durch Interessenklärung, kooperative Lösungsentwicklung und die Orientierung an objektiven Kriterien entstehen. Gleichzeitig wird deutlich, dass Kommunikationsstörungen, emotionale Dynamiken und unterschiedliche Verhandlungs- und Persönlichkeitstypen häufige Ursache für Konflikte in Verhandlung darstellen.
Modelle wie Axiome von Watzlawick, das Riemann-Thomann-Kreuz sowie die Typisierung von Verhandlungsstillen bieten wertvolle Hilfsmittel, um Gesprächsverlaufe besser zu verstehen und das eigene Verhalten situationsgerecht anzupassen. Insgesamt zeigt die Hausarbeit, dass erfolgreiche Verhandlungsführung erlernbar ist und maßgeblich von Reflexionsfähigkeit, Flexibilität und der bewussten Ausrichtung auf den jeweiligen Verhandlungspartner abhängt.\\

Abschließend möchten wir noch auf unsere Zusammenarbeit bei der Erstellung dieser Hausarbeit eingehen. Wir haben eng zusammengearbeitet und regelmäßig miteinander kommuniziert. Die inhaltlichen Aufgaben wurden zwischen uns aufgeteilt, aber koordiniert. Die Arbeitslast war gleichmäßig verteilt: Carina hat etwa 50 \% der Arbeit übernommen und Paloma ebenfalls etwa 50 \%.