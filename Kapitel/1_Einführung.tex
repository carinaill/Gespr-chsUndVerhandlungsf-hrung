\chapter{Einführung}

In der vorliegenden Hausarbeit wird die Gesprächs- und Verhandlungsführung einer konkreten Teamsituation analysiert. 
Ziel ist es, Probleme im Gesprächsverlauf zu identifizieren, das Verhalten und die Formulierungen der Beteiligten auszuwerten
und daraus Schlussfolgerungen für eine bessere Vorbereitung und Gesprächsführung abzuleiten.\\
Darüber hinaus werden die Persönlichkeitstypen der Beteiligten nach dem Riemann-Thomann-Modell betrachtet, 
um Empfehlungen für die individuelle Ansprache und für die Teamassistentinnen zu formulieren.
Ein weiterer Fokus liegt auf der Erkennung von Verhandlungstypen, deren Einfluss auf das eigene Verhalten 
sowie auf die Anwendung sachgerechter Verhandlungsprinzipien nach dem Harvardkonzept, 
um zu einer konstruktiven und lösungsorientierten Gesprächsführung zu gelangen.\\
Die Hausarbeit verbindet damit theoretische Grundlagen aus Kommunikation, Gesprächsführung und Verhandlungslehre 
mit praktischen Empfehlungen für die konkrete Verhandlungssituation. \cite{troczynski2023} \cite{vanderwijst2020} \cite{watzlawick_axiome}