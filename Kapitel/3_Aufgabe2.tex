\chapter{Aktives Zuhören}
Das aktive Zuhören ist "eine Kommunikationstechnik mit dem Ziel, eine 
positive Kommunikationsatmosphäre zu schaffen"\footnote{Buch mit 2 am Ende von Jung und Krebs}. Das geschieht mit Hilfe verbaler Signale,
wie zustimmende "hm" Geräusche oder Nachfragen wie "Ach echt?" oder "Und was ist dann passiert?"
sowie nonverbaler Signale wie Nicken, Körperhaltung und Augenkontakt.

\section{a: Positive Effekte bei bewusster Anwendung in Verhandlungssituationen}
Dieses Verhalten kann positive Effekte auf den Verhandlungsparter haben, da er sich verstanden und gehört fühlt.
Dadurch kann sich der Verhandlungsüpartner sich wohlfühlen und unbewusste Anspannungen ablegen. 
Insofern hat bewusstes Anwenden von aktivem Zuhören eine positve Auswirkung auf das generelle 
Verhandlungsklima. Viele Menschen empfinden es als angenehm wenn ihnen aktiv zugehört wird, 
in Bezug auf Verhandlungen kann sich dadurch also die Beziehungsebene positiv verändern, wodurch 
der Verhandlungspartner möglicherweise eher gewillt ist Kmpromisse einzugehen oder sich anzupassen.
Außerdem könnte er für zukünftige Verhandlungen in Zukunft eher auf "uns" zurückkommen, da ihm die freundliche Art
unterbewusst in Erinnerung geblieben ist. 

\section{b: Elemente für die Gesprächsführung}
Das Nicken und der Augenkontakt können genutzt werden, um dem Verhandlungspartner unterbewusst ein positives
Gefühl zu vermitteln. Nachfragen wie "Das hat er gesagt?" können dazu dienen die Beziehungsebene bewusst zu erhöhen.
Der Vertragspartner bekommt das Gefühl dass man auf "seiner Seite" ist. Fragen wie "Und dann?" können natürlich
auch der Informationsentnahme dienen.