{\let\clearpage\relax\vspace{3\baselineskip}
\chapter{Schwierigkeiten und Konflikte im Verhandlungsgespräch}}
\section{a: Konflikte und deren Ursachen}
Im Verlauf eines Verhandlungsgespräches können unterschiedliche Schwierigkeiten und Konflikte auftreten,
die den Gesprächsprozess beeinträchtigen. 
Ein zentraler Konfliktbereich ist die Kommunikation zwischen den Parteien. 
Missverständnisse, unklare Aussagen oder unterschiedlich interpretierte nonverbale Signale können dazu führen, dass sich Beteiligte nicht ernst
genommen fühlen und emotional reagieren.\footnote{Vgl. \cite{FisherUryPatton1991}, Kapitel 2 „Separate the People from the Problem“, S.13-23.} \\
Diese Kommunikationsprobleme lassen sich häufig auf eine unzureichende Umsetzung des Gesprächsleitfadens zurückführen.\footnote{\cite{gesprächsleitfaden}} 
Bereits in der Eröffnungsphase\footnote{\cite{gesprächsleitfaden}, Phase 0}  können Konflikte entstehen, wenn es nicht gelingt, eine konstruktive Atmosphäre zu schaffen, 
wird beispielsweise auf die falschen Verhandlungstypen eingegangen.\\
Weitere Schwierigkeiten zeigen sich in der Phase der Agenda-Festlegung\footnote{\cite{gesprächsleitfaden}, Phase 1}. 
Werden Themen und Ziele nicht gemeinsam geklärt, entstehen unterschiedliche Erwartungen an den Gesprächsverlauf.
Dies kann dazu führen, dass sich einzelne Parteien übergangen fühlen oder zentrale Anliegen im weiteren Verlauf nicht berücksichtigt werden.\\
In der konfrontativen und kompetitiven Phase\footnote{\cite{gesprächsleitfaden}, Phasen 2 und 3} treten Konflikte häufig durch starre Positionsvertretung, 
emotionale Argumentation oder fehlendes aktives Zuhören auf. Anstatt auf Argumente der Gegenseite einzugehen, wird versucht, 
die eigene Position zu verteidigen oder durchzusetzen. Dadurch kann sich der Konflikt verschärfen und eine Eskalationsdynamik\footnote{Vgl. \cite{watzlawick_axiome}} entstehen.\\
Besonders problematisch ist es, wenn der Übergang in die kooperative Phase\footnote{\cite{gesprächsleitfaden}, Phase 4} nicht gelingt. Bleiben die Parteien in der 
Wettbewerbslogik verhaftet, werden Interessen nicht offengelegt und kreative Lösungsansätze nicht entwickelt. Die Ursachen hierfür
liegen meist in mangelnder Flexibilität im Umgang mit dem Leitfaden oder fehlender Gesprächsstruktur.

\section{b: Lösungsansätze}
Ein Leitfaden soll Orientierung geben, darf aber nicht als starres Regelwerk verstanden werden. 
Beide Vertragsparteien sollten bereit sein, die Struktur an die Situation anzupassen, 
indem weniger relevante Punkte übersprungen oder Themen verschoben werden. 
So bleibt der Leitfaden nützlich, ohne den Gesprächsfluss zu blockieren.\\
Missverständnisse lassen sich durch aktives Zuhören, Nachfragen und Zusammenfassen von Aussagen reduzieren.\footnote{Vgl. \cite{FisherUryPatton1991}, S. 21.} 
Es hilft, die eigenen Gedanken klar zu formulieren und regelmäßig zu prüfen, ob die andere Seite die Aussagen richtig versteht. 
Auch nonverbale Signale sollten bewusst wahrgenommen werden, um versteckte Spannungen früh zu erkennen.\\
Ein kurzer Einstieg, in dem beide Seiten ihre Erwartungen an die Verhandlung äußern, kann Missverständnisse vermeiden. 
Außerdem ist es hilfreich, Positionen von dahinterliegenden Interessen zu trennen, um kreative Lösungen zu ermöglichen.\footnote{Vgl. \cite{FisherUryPatton1991}, S. 23.} \\
Außerdem kann es bei starken Konflikten helfen, eine dritte Partei hinzuzuschalten\footnote{Vgl. \cite{krebsjung2016}, S.361.}, diese kann neutraler auf die Geschehnisse blicken.
Wenn es weiterhin unüberwindbare Konflikte gibt, lohnt es sich die niederlagelose Konfliktbewältigungsstrategie \footnote{Vgl. \cite{kreggenfeld2021}, S.41.} anzueignen.
