\chapter{Schwierigkeiten und Konflikte im Verhandlungsgespräch}
\section{Konflikte und deren Ursachen}

Im Verlauf eines Verhandlungsgespräches können unterschiedliche Schwierigkeiten und Konflikte auftreten,
die den Gesprächsprozess stören oder blockieren können. 
Ein zentraler Bereich, in dem solche Probleme entstehen, ist die Kommunikation zwischen den Parteien. 
Wenn Aussagen missverstanden werden oder nonverbale Signale unterschiedlich interpretiert werden, 
kann das Gespräch schnell in eine Abwärtsspirale geraten, weil sich Gesprächspartner angegriffen oder nicht gehört fühlen. 
Dies erklärt auch, warum in der Verhandlungsliteratur immer wieder betont wird, wie wichtig aktives Zuhören und klare Ausdrucksweise sind, 
um Kommunikationsstörungen zu vermeiden.\footnote{} 

Eng mit der Kommunikation hängt die Gesprächsführung und der eingesetzte Leitfaden zusammen. 
Laut Jung und Krebs ist ein strukturierter Gesprächsleitfaden ein wichtiges Hilfsmittel, 
um einen Verhandlungsprozess zu planen und Schritt für Schritt durchzuführen.\footnote{Vgl. \cite{krebsjung2016}} 
Gleichzeitig kann dieser Leitfaden selbst zur Konfliktquelle werden, wenn er zu starr genutzt wird: Gesprächspartner fühlen sich eingeengt, 
die Dynamik des Austauschs leidet und andere relevante Themen bleiben unberücksichtigt.
Deshalb betonen Jung und Krebs, dass der Leitfaden zwar Orientierung geben, aber nicht jede spontane Entwicklung im Gespräch blockieren sollte.\footnote{\cite{krebsjung2016}}
Kommt es hier zu Problemen – etwa weil der Leitfaden zu starr oder zu unflexibel umgesetzt wird –, 
fühlen sich die Parteien eingeengt oder übergangen. 
Gleichzeitig kann ein zu offener Leitfaden dazu führen, dass das Gespräch ins Stocken gerät, weil keine klare Orientierung vorhanden ist.\footnote{\cite{krebsjung2016}}

Schließlich können externe Faktoren, wie Zeitdruck, asymmetrische Informationen oder schon bestehende Vorurteile,
den Verlauf einer Verhandlung belasten und Konflikte verstärken. 
Diese Faktoren wirken oft subtil, sind aber nach Jung und Krebs Teil dessen, was Verhandlungen komplex macht, 
weil sie zusätzliche Ebenen von Unsicherheit und Risiko aufbauen.\footnote{\cite{krebsjung2016}}

Die Ursachen für Konflikte liegen also meist nicht nur in der Persönlichkeit oder den Interessen der Parteien, 
sondern häufig in der Gestaltung und Anwendung des Gesprächsleitfadens: Zu starre Leitfäden, unklare Strukturierung der Themen, 
fehlende Priorisierung und mangelnde Anpassung an die Gesprächsdynamik begünstigen Missverständnisse, Frustration und Blockaden.
Ein Grund kann sein, dass die Parteien unterschiedliche Erwartungen an den Ablauf der Verhandlung haben. Zum Beispiel denkt eine Seite, 
dass man sich strikt an den Leitfaden hält, während die andere eher flexibel diskutieren möchte. 
Diese Diskrepanz kann schnell zu Frustration führen. Außerdem können unterschiedliche Prioritäten der Parteien – etwa kurzfristige vs. 
langfristige Ziele – dazu führen, dass bestimmte Themen zu stark oder zu wenig Gewicht bekommen.

\section{Lösungsansätze}

1. Flexibler Umgang mit dem Gesprächsleitfaden:
Ein Leitfaden soll Orientierung geben, darf aber nicht als starres Regelwerk verstanden werden. 
Beide Vertragsparteien sollten bereit sein, die Struktur an die Situation anzupassen, 
indem weniger relevante Punkte übersprungen oder Themen verschoben werden. 
So bleibt der Leitfaden nützlich, ohne den Gesprächsfluss zu blockieren.\\
2. Klare und offene Kommunikation:
Missverständnisse lassen sich durch aktives Zuhören, Nachfragen und Zusammenfassen von Aussagen reduzieren.\footnote{} 
Es hilft, die eigenen Gedanken klar zu formulieren und regelmäßig zu prüfen, ob die andere Seite die Aussagen richtig versteht. 
Auch nonverbale Signale sollten bewusst wahrgenommen werden, um versteckte Spannungen früh zu erkennen.\\
3. Gemeinsames Abklären von Erwartungen und Interessen:
Konflikte entstehen oft durch unterschiedliche Vorstellungen vom Ablauf oder Prioritäten. 
Ein kurzer Einstieg, in dem beide Seiten ihre Erwartungen an die Verhandlung äußern, kann Missverständnisse vermeiden. 
Außerdem ist es hilfreich, Positionen von dahinterliegenden Interessen zu trennen, um kreative Lösungen zu ermöglichen.\\
4. Umgang mit Emotionen:
Emotionale Spannungen lassen sich durch eine respektvolle Gesprächsatmosphäre, Pausen bei Eskalationen und neutrale Formulierungen reduzieren.
Es kann auch sinnvoll sein, problematische Punkte zunächst sachlich zu sammeln und erst später auf Lösungen hinzuarbeiten\\
5. Reflexion und Feedback während der Verhandlung:
Regelmäßige kurze Pausen oder Check-ins ermöglichen es den Parteien, den Gesprächsverlauf zu reflektieren, Missverständnisse zu klären und den Leitfaden bei Bedarf anzupassen.\\
Zusammenfassend:
Konflikte lassen sich also vermeiden oder entschärfen, wenn der Gesprächsleitfaden flexibel eingesetzt wird, 
die Kommunikation klar und aktiv gestaltet wird, Erwartungen transparent gemacht werden und Emotionen bewusst 
berücksichtigt werden. Durch diese Ansätze wird nicht nur die Verhandlung effektiver, sondern auch die 
Zusammenarbeit zwischen den Parteien gestärkt.