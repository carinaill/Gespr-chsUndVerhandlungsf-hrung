\chapter{Schwierigkeiten und Konflikte im Verhandlungsgespräch}
\section{Konflikte und deren Ursachen}

Im Verlauf eines Verhandlungsgespräches können unterschiedliche Schwierigkeiten und Konflikte auftreten,
die den Gesprächsprozess stören oder blockieren können. 
Ein zentraler Bereich, in dem solche Probleme entstehen, ist die Kommunikation zwischen den Parteien. 
Wenn Aussagen missverstanden werden oder nonverbale Signale unterschiedlich interpretiert werden, 
kann das Gespräch schnell in eine Abwärtsspirale geraten, weil sich Gesprächspartner angegriffen oder nicht gehört fühlen.
\footnote{Vgl. \cite{FisherUryPatton1991}, Kapitel 2 „Separate the People from the Problem“, S.13-23.} \\
Eng mit der Kommunikation hängt die Gesprächsführung und der eingesetzte Leitfaden zusammen. 
Ein strukturierter Gesprächsleitfaden ist ein wichtiges Hilfsmittel, 
um einen Verhandlungsprozess zu planen und Schritt für Schritt durchzuführen.
Gleichzeitig kann dieser Leitfaden selbst zur Konfliktquelle werden, wenn er zu starr genutzt wird, wodurch
die Dynamik des Austauschs leideen und andere relevante Themen unberücksichtigt bleiben können.
Deshalb dollte der Leitfaden zwar Orientierung geben, aber nicht jede spontane Entwicklung im Gespräch blockieren, damit sich die Parteien nicht eingeengt oder übergangen fühlen. 
Gleichzeitig kann ein zu offener Leitfaden dazu führen, dass das Gespräch ins Stocken gerät, weil keine klare Orientierung vorhanden ist.\\
Die Ursachen für Konflikte liegen also in unklarer Strukturierung der Themen, 
unterschiedlichen Erwartungen und mangelnder Anpassung an die Gesprächsdynamik. Zum Beispiel denkt eine Seite, 
dass man sich strikt an den Leitfaden hält, während die andere eher flexibel diskutieren möchte. 
Diese Diskrepanz kann schnell zu Frustration führen. Außerdem können unterschiedliche Prioritäten der Parteien dazu führen, dass bestimmte Themen zu stark oder zu wenig Gewicht bekommen.

\section{Lösungsansätze}

Ein Leitfaden soll Orientierung geben, darf aber nicht als starres Regelwerk verstanden werden. 
Beide Vertragsparteien sollten bereit sein, die Struktur an die Situation anzupassen, 
indem weniger relevante Punkte übersprungen oder Themen verschoben werden. 
So bleibt der Leitfaden nützlich, ohne den Gesprächsfluss zu blockieren.\\
Missverständnisse lassen sich durch aktives Zuhören, Nachfragen und Zusammenfassen von Aussagen reduzieren.\footnote{Vgl. \cite{FisherUryPatton1991}, S. 21.} 
Es hilft, die eigenen Gedanken klar zu formulieren und regelmäßig zu prüfen, ob die andere Seite die Aussagen richtig versteht. 
Auch nonverbale Signale sollten bewusst wahrgenommen werden, um versteckte Spannungen früh zu erkennen.\\
Konflikte entstehen oft durch unterschiedliche Vorstellungen vom Ablauf oder Prioritäten. 
Ein kurzer Einstieg, in dem beide Seiten ihre Erwartungen an die Verhandlung äußern, kann Missverständnisse vermeiden. 
Außerdem ist es hilfreich, Positionen von dahinterliegenden Interessen zu trennen, um kreative Lösungen zu ermöglichen.\\
Außerdem kann es bei starken Konflikten helfen, eine dritte Partei hinzuzuschalten\footnote{Vgl. \cite{krebsjung2016}, S.361.}, diese kann neutraler auf die Geschehnisse blicken.
Wenn es weiterhin unüberwindbare Konflikte gibt, lohnt es sich die niederlagelose Konfliktbewältigungsstrategie \footnote{Vgl. \cite{kreggenfeld2021}, S.41.} anzueignen.