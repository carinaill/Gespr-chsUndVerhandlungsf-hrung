{\let\clearpage\relax\vspace{3\baselineskip}
\chapter{Anwendung des Harvardkonzepts}}
\textbf{Harvard Prinzip 1: Trennung von Sach- und Beziehungsebene}\\
Um insbesondere Herrn Mahler als näheorientierten Typen auf der Beziehungsebene abzuholen und gleichzeitig für Herrn Köhler sachlich zu bleiben, könnten Frau Schneider und Frau Klar formulieren: 
„Uns geht es nicht um Kritik an einzelnen Personen oder Abteilungen, sondern um die Arbeitsprozesse, die sich seit der Umstrukturierung als schwierig erwiesen haben.“  
Diese Aussage signalisiert Respekt und Kooperationsbereitschaft und verhindert, dass sich die Vorgesetzten persönlich angegriffen fühlen.

\textbf{Harvard Prinzip 2: Fokus auf Interessen statt Positionen}\\
Statt einzelne Probleme aufzuzählen, könnten die Teamassistentinnen ihr zentrales Interesse klar benennen, zum Beispiel: 
„Unser Ziel ist ein verlässlicher Informationsfluss und klar geregelte Zuständigkeiten, damit wir unsere Arbeit effizient und fehlerfrei erledigen können.“  
Damit sprechen sie Herrn Köhler als distanz-dauerorientierten Typen sachlich an, während Herr Mahler den kooperativen Grundgedanken erkennt.

\textbf{Harvard Prinzip 3: Entwicklung von Optionen statt vorschneller Bewertungen}\\
Um beide Vorgesetzte aktiv einzubinden, könnten Frau Schneider und Frau Klar vorschlagen: 
„Vielleicht können wir gemeinsam überlegen, welche Anpassungen an der Schnittstelle zum Service Center sinnvoll wären, zum Beispiel durch einheitliche Standards oder eine zeitlich begrenzte Testphase.“  
Dies fördert eine kooperative Atmosphäre und ermöglicht allen Teilnehmenden Beispiel gestützt und lösungsorientiert mitzuwirken.

\textbf{Harvard Prinzip 4: Nutzung objektiver Kriterien}\\
Um insbesondere Herrn Köhlers Bedürfnis nach Struktur und Kontrolle entgegenzukommen, könnten objektive Maßstäbe vorgeschlagen werden: 
„Zur Bewertung möglicher Lösungen könnten wir uns an Kriterien wie Bearbeitungszeiten oder der Anzahl von Rückfragen orientieren.“  
Objektive Kriterien reduzieren emotionale Spannungen und erleichtern eine sachliche Entscheidungsfindung für beide Seiten.