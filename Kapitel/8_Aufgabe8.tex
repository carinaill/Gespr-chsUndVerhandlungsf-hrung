\chapter{Anwendung des Harvardkonzepts}
Sachgerechtes Verhandeln nach dem Harvard-Konzept zielt darauf ab, Konflikte kooperativ und lösungsorientiert zu bearbeiten. Im Mittelpunkt stehen dabei Interessen, gemeinsame Lösungsoptionen und objektive Kriterien statt Positionen oder Schuldzuweisungen.\\
Harvard Prinzip 1: Trennung von Sach- und Beziehungsebene
Um die Beziehungsebene zu entlasten, könnten Frau Schneider und Frau klar beispielsweise sagen: “Uns geht es nicht um Kritik an einzelnen Personen, sondern um die Arbeitsprozesse, die sich seit der Umstrukturierung als schwierig erwiesen haben.“ 
Diese Aussage entlastet die Beziehungsebene und vermeidet Vorwürfe signalisiert Respekt sowie Kooperationsbereitschaft.\\
Harvard Prinzip 2: Fokus auf Interessen statt Positionen
Statt Beschwerden zu äußern, könnten die Assistentinnen ihr zentrales Interesse formulieren, etwa: „Unser Anliegen ist ein reibungsloser Informationsfluss und klare Zuständigkeiten, damit wir effizient arbeiten können.“
Dadurch wird für Herrn Köhler und Herrn Mahler deutlich, worum es den Assistentinnen tatsächlich geht, ohne dass sie sich angegriffen fühlen.\\
Harvard Prinzip 3: Entwicklung von Optionen statt vorschneller Bewertungen 
Zur gemeinsamen Lösungsfindung könnten sie vorschlagen: “Vielleicht können wir gemeinsam überlegen, wie wir die Schnittstelle zum Service Center verbessert werden kann, zum Beispiel durch klare Standards oder eine Testphase.”
Diese Aussage lädt die Vorgesetzten aktiv zur Mitgestaltung ein und fördert eine kooperative Verhandlung.\\
Harvard Prinzip 4: Nutzung objektiver Kriterien 
Zur sachlichen Bewertung möglicher Lösungen könnten Frau Schneider und Frau Klar anmerken: „Wir könnten uns bei der Entscheidung an objektiven Kriterien wie Bearbeitungszeiten oder der Anzahl von Rückfragen orientieren.“ 
Objektive Maßstäbe reduzieren emotionale Spannungen und erleichtern eine faire Entscheidungsfindung.