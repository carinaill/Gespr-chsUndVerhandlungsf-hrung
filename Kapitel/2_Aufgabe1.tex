% ----------------------------------------------------------
% ----------------------------------------------------------
\chapter{Verhandlungen mit einem potentiellen Lieferanten}

\section{a: Strategische Wahl des Gesprächsortes}
\label{sec:gesprächsortswahl}
\cite{}

Der Ort dient nicht lediglich als logistischer Rahmen, sondern beeinflusst den Verhandlungsprozess auf psychologischer, sozialer und strategischer Ebene. 
Die Verhandlungsforschung zeigt, dass Kontext- und Umweltfaktoren die Wahrnehmung, das Stressniveau sowie das Verhalten der Verhandlungspartner beeinflussen können.\footnote{}

Ein zentrales Kriterium bei der Wahl des Verhandlungsortes ist die Frage nach Macht- und Kontrollverhältnissen. 
Van der Wijst et al. verweisen auf den sogenannten Home-Field Advantage, der beschreibt, dass Akteure an ihrem eigenen Standort ein höheres Maß an Komfort, Kontrolle und Selbstvertrauen empfinden.\footnote{}
Diese psychologischen Effekte können sich indirekt auf das Verhandlungsverhalten auswirken, etwa durch ein souveräneres Auftreten oder eine stärkere Durchsetzungsfähigkeit.
Für Verhandlungen mit einem neuen Lieferanten bedeutet dies, dass ein Treffen am eigenen Unternehmensstandort strategisch sinnvoll sein kann, wenn eine starke Verhandlungsposition signalisiert werden soll. 
Gleichzeitig ist zu berücksichtigen, dass ein zu ausgeprägter Heimvorteil auch als Machtdemonstration wahrgenommen werden kann und die Beziehungsebene belasten könnte.

!! Ein neutraler Ort wirkt oft moderater und gleichberechtigter, was Vertrauen schaffen kann.
Er senkt das Gefühl, „unter Druck gesetzt“ zu werden und ist besonders bei erstmaligen Verhandlungen sinnvoll.
Praktische und logistische Aspekte
Reisekosten und -dauer
Komfort und Professionalität (z. B. Meetingräume statt Cafés)
Vorbereitungsmöglichkeiten (Whiteboard, Präsentationstechnik)

Ein weiteres zentrales Kriterium stellt die psychologische Wirkung der Verhandlungsumgebung dar.\footnote{}
Insbesondere natürliche oder entspannte Umgebungen können Stress reduzieren und positive Emotionen fördern.
Im Kontext einer potenziellen neuen Lieferantenbeziehung, die häufig auf langfristige Zusammenarbeit abzielt, 
kann daher ein neutraler oder atmosphärisch angenehmer Ort sinnvoll sein, um kooperative Lösungsansätze zu fördern.

Ein weiterer Aspekt betrifft die Wahrnehmung des Verhandlungsortes hinsichtlich Professionalität und Seriosität. 
Die Studie von van der Wijst et al. zeigt, dass informelle oder ungewöhnliche Umgebungen zwar als entspannter wahrgenommen werden, jedoch zugleich als weniger professionell gelten können.\footnote{}
Gerade bei Erstkontakten mit neuen Lieferanten kann die wahrgenommene Professionalität eine wichtige Rolle für Vertrauen und Verbindlichkeit spielen.

%Fußnoten
%Vgl. van der Wijst, Hong \and Damen (2020), Introduction.
%gl. van der Wijst, Hong \and Damen (2020), Abschnitt „Home-Field Advantage“.
%Vgl. ebd., Abschnitt „Context and Environment“.
%Vgl. ebd., Ergebnisse der empirischen Studie.
%Vgl. ebd., Unterscheidung distributiver und integrativer Verhandlungen.
%Vgl. ebd., Diskussion der Wahrnehmung von Professionalität.
%Vgl. ebd., Discussion and Conclusion.

\section{b: Informierung zur Terminvorbereitung}
\label{sec:terminvorbereitung}


%Kreggenfeld:
Zuerst einmal mache ich  mir Gedanken über den Verhandlungsgegenstand, also die Rahmenbedingungen der Lieferungen.
Dafür sollten durchschnittliche Preise für Lieferungen dieser Branche hinzugezogen werden. 
Weitere Informationen wie durchschnittliche Lieferzeiten und weitere Konditionen sollten außerdem in Erfahrung gebracht werden.
Es ist wichtig, sich über mögliche alternative Lieferanten zu informieren, um die Ersetzbarkeit des Verhandlungsgegenstandes 
besser einschätzen zu können und somit entspannter in die Verhandlung gehen zu können, sollten Alternativen exisitieren.
Ich sollte außerdem mit der zugehörigen Abteilung meines Unternehmens klären, wie viel Verhandlungsspiellraum ich habe,
also wie weit ich dem Lieferanten entgegenkommen darf, bzw auf welche Aspekte auf keinen Fall verzichtet werden können.
Dafür sollte auch die Prioritäten in Erfahrung bringen, um meine Ziele in Muss-, Soll-, und Kann-, Ziele unterteilen zu können.
Aus der umfangreichen Recherche über meinen Lieferanten kann ich evtl. auch seine Ziele und die dahinterstehenden Interessen ein wenig
besser einschätzen und auch diese kategorisieren. 
Diese Kategorisierungen kann ich dann heranziehen um Gemeinsamkeiten und Unterschiede zwischen unseren Interessen herauszuarbeiten.
Daraus kann ich dann mögliche Lösungsansätze ableiten und besser einschätzen an welchem Punkt ich die Verhandlungen starten lasse.
Ich kann mich außerdem informieren ob es über den Lieferanten bereits bekannte Muster gibt, dafür kann ich Beispielweise linkedIn oder Kununu nutzen.
Ich kann mich auch Informieren, inwiefern der Lieferant von der angestrebten Kooperation profetiert, um die Machtverhältnisse planen zu können.

Anhand dieser ganzen Informationen  mache ich  mir Gedanken darüber, ob ich eine Ergebnisorientiertes oder ein Beziehungsorientiertes ergebnis anstrebe.
Dafür ziehe ich Informationen zu (prozessorientierten) Verhandlungsstrategien wie der Win-Win Strategie heran.
Ich mache mir Gedanken über meine Interessen und recherchiere meinen Lieferanten umfänglich, dazu könnte ich seine Webseite und seine eventuelle Präsenz
in den sozialen Medien nutzen.
Anhand der Ergebnisse meiner Interessen und des Auftretens des Lieferandes entscheide ich mich dann für eine der recherchierten Verhandlungsstrategien
Man sollte sich also bestmöglich vorbereiten, darf dabei aber nicht die Spontanität verlieren die sich im Laufe der Verhandlung ergeben werden.

Ich recherchiere mögliche Pläne B
Des Weiteren ziehe ich die Ergebnisse aus 1a hinzu um zu entscheiden welcher Ort für diese Verhandlung angemessen ist.
Außerdem recherchiere ich wie lang ein Verhandlungsgespräch in meiner Branche durchschnittlich dauert und wann die beste Uhrzeit dafür ist.
