\chapter{„Kuchen vergrößern“}
\cite{}
\label{sec:approach}
\section{Im Rahmen sachgerechten verhandelns}
„Kuchen vergrößern“ stellt eine Win-Win Verhandlungsstrategie dar\footnote{Vgl. \cite{kreggenfeld2021}, S.203.}, bei
der es darum geht, den Vertragskuchen zu vergrößern.
Der Vertragskuchen sind dabei die zur Verfügung stehenden Ressourcen, also der Summe aller Vor- und Nachteile,
die beide Parteien aus der Einigung erzielen können. \footnote{Vgl. \cite{krebsjung2016}, S.279.} 
Diese zu vergrößern, heißt dass alle Vertragsparteien ihre verfolgten Ziele erreichen. \\
Die Vergrößerung des Kuchens setzt voraus, dass sich Verhandler von der Annahme eines unveränderlichen fixed pie lösen.\footnote{Vgl. \cite{krebsjung2016}, S.280.} 
Diese Annahme beruht häufig auf kognitiven Verzerrungen wie dem incompatibility bias,\footnote{Vgl. \cite{krebsjung2016}, S.280.} wonach Interessen der Parteien als zwingend gegensätzlich wahrgenommen werden. 
Tatsächlich bestehen jedoch vielfach Spielräume zur Wertschöpfung, etwa durch die Einbeziehung weiterer Verhandlungsdimensionen neben dem Preis, wie Lieferfristen, Zusatzleistungen oder Qualität. \footnote{Vgl. \cite{krebsjung2016}, S. 279.} \\
Im Ergebnis bedeutet das „Kuchen vergrößern“ im sachgerechten Verhandeln, 
den Fokus von einer rein distributiven Verteilung hin zu einer gemeinsamen Wertschöpfung zu verschieben, 
wobei die Grenze des Kartellverbots stets zu beachten ist.

\section{Sinnvolle Situationen}
Häufig gehen Verhandlungspartner vorschnell davon aus, dass ein Vorteil für die eine Seite automatisch einen Nachteil für die andere bedeutet. 
In solchen Fällen lohnt es sich, die Verhandlung bewusst weiter zu denken.\\
Ein typischer Fall dafür ist die Preisverhandlung, bei der sich zwei Parteien nicht einigen können. 
Statt weiter ausschließlich über den Preis zu verhandeln, können zusätzliche Punkte einbezogen werden, etwa längere Zahlungsziele, Serviceleistungen oder Garantien. 
Dadurch entsteht für beide Seiten ein Mehrwert, ohne dass eine Partei spürbar „nachgeben“ muss.
Auch unterschiedlicher Zeitdruck kann Anlass sein, den Verhandlungskuchen zu vergrößern. 
Benötigt eine Partei schnell eine Einigung, während die andere zeitlich flexibler ist, kann dies ausgeglichen werden, etwa durch eine frühere Lieferung oder eine beschleunigte Entscheidungsfindung anstelle eines Preisnachlasses.\\
Sinnvoll ist der Ansatz außerdem, wenn eine Verhandlung festgefahren wirkt, obwohl beide Seiten grundsätzlich an einer Einigung interessiert sind. 
Häufig liegt die Blockade dann weniger an unvereinbaren Interessen als daran, dass nur über einen einzelnen Punkt gestritten wird.
Wird der Verhandlungsgegenstand erweitert, zum Beispiel durch die Aufnahme einer Verlängerungsoption oder eines Zusatzauftrags, kann neue Bewegung in die Gespräche kommen.\\
Ein weiteres Beispiel sind Verhandlungen, in denen die Parteien unterschiedliche Erwartungen an die Zukunft haben. 
Glaubt eine Seite an eine positive Entwicklung, während die andere skeptisch ist, können flexible Lösungen wie erfolgsabhängige Vergütungen oder spätere Anpassungsklauseln beiden Sichtweisen gerecht werden.\\
Besonders sinnvoll ist die Suche nach einer Vergrößerung des Verhandlungskuchens, wenn die jeweilige BATNA (Best Alternative to a Negotiated Agreement) für beide Seiten wenig attraktiv oder mit erheblichen Risiken verbunden ist. 
Ist die Alternative zum Verhandlungsergebnis beispielsweise ein kostspieliger Rechtsstreit, ein Projektabbruch oder der Verlust eines Geschäftspartners,
steigt der Anreiz, innerhalb der Verhandlung zusätzliche Lösungsoptionen zu entwickeln. 
Die gemeinsame Erweiterung des Verhandlungsgegenstandes kann in solchen Fällen dazu beitragen, Eskalations-, Durchsetzungs- oder Folgekosten zu vermeiden und dennoch ein für beide Seiten tragfähiges Ergebnis zu erzielen.\\
Immer dann, wenn sich eine Verhandlung wie ein Entweder-oder anfühlt, lohnt es sich zu prüfen, ob man sie in ein Sowohl-als-auch verwandeln kann.