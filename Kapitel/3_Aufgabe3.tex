\chapter{„Kuchen vergrößern“}
\label{sec:approach}
\section{Im Rahmen sachgerechten verhandelns}
„Kuchen vergrößern“ stellt eine Win-Win Verhandlungsstrategie dar\footnote{Vgl. \cite{kreggenfeld2021}, S.203.}, bei
der es darum geht, den Vertragskuchen zu vergrößern.
Der Vertragskuchen sind dabei die zur Verfügung stehenden Ressourcen, also der Summe aller Vor- und Nachteile,
die beide Parteien aus der Einigung erzielen können. \footnote{Vgl. \cite{krebsjung2016}, S.279.} 
Diese zu vergrößern, heißt dass alle Vertragsparteien ihre verfolgten Ziele erreichen. \\
Die Vergrößerung des Kuchens setzt voraus, dass sich Verhandler von der Annahme eines unveränderlichen fixed pie lösen.\footnote{Vgl. \cite{krebsjung2016}, S.280.} 
Diese Annahme beruht häufig auf kognitiven Verzerrungen wie dem incompatibility bias,\footnote{Vgl. \cite{krebsjung2016}, S.280.} wonach Interessen der Parteien als zwingend gegensätzlich wahrgenommen werden. 
Tatsächlich bestehen jedoch vielfach Spielräume zur Wertschöpfung, etwa durch die Einbeziehung weiterer Verhandlungsdimensionen neben dem Preis, wie Lieferfristen, Zusatzleistungen oder Qualität. \footnote{Vgl. \cite{krebsjung2016}, S. 279.} \\

\section{Sinnvolle Situationen}
Es ist sinnvoll, nach Möglichkeiten zu suchen, den „Verhandlungskuchen“ zu vergrößern, 
wenn eine Verhandlung festgefahren erscheint oder sich auf nur einen Punkt konzentriert. 
Beispielsweise können in Preisverhandlungen zusätzliche Aspekte wie längere Zahlungsziele, 
Serviceleistungen oder Garantien einbezogen werden, sodass beide Seiten profitieren, 
ohne dass jemand spürbar nachgeben muss. 
Auch unterschiedlicher Zeitdruck kann genutzt werden: Muss eine Partei schnell eine Einigung erreichen, 
während die andere flexibler ist, können frühere Lieferungen oder beschleunigte Entscheidungen zusätzlichen Wert schaffen. 
Unterschiedliche Erwartungen an die Zukunft lassen sich durch flexible Lösungen wie erfolgsabhängige Vergütungen oder Anpassungsklauseln ausgleichen. 
Besonders relevant ist diese Strategie, wenn die jeweiligen Alternativen zur Verhandlung (BATNA) wenig attraktiv oder riskant sind, 
etwa ein Rechtsstreit, Projektabbruch oder der Verlust eines Geschäftspartners. 
In solchen Situationen kann die gemeinsame Erweiterung des Verhandlungsgegenstandes dazu beitragen, 
ein für beide Seiten tragfähiges Ergebnis zu erzielen und zusätzliche Kosten zu vermeiden. 
Generell lohnt es sich immer, zu prüfen, ob ein scheinbares Entweder-oder in ein Sowohl-als-auch transformiert werden kann, 
um Mehrwert für beide Parteien zu schaffen.
