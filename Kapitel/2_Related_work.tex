% ----------------------------------------------------------
% ----------------------------------------------------------
\chapter{Related Work}

Your related work goes here...



% ----------------------------------------------------------
\section{Visual Analytics}
\label{sec:visual_analytics}


Figure \ref{fig:visualanalytics1} shows ...

\begin{figure}[H]
	\begin{centering}
		\includegraphics[width=130mm]{Bilder/Keim_vad.png}
		\caption{The Scope of Visual Analytics \cite{Keim.0507July2006}}
		\label{fig:visualanalytics1}
	\end{centering}
\end{figure}

\begin{figure}[H]
	\begin{centering}
		\includegraphics[width=130mm]{Bilder/Keim_vad3.png}
		\caption{The Visual Analytics Process \cite{Keim.2010}}
		\label{fig:visualanalytics2}
	\end{centering}
\end{figure}


% ----------------------------------------------------------
\section{Non-linear Projections}
\label{sec:nonlinear_projections}

\cite{Sammon.1969}\\
\cite{Demartines.1997}\\
Isomap \cite{Tenenbaum.2000}\\
SNE \cite{Hinton.2002}\\
Laplacian Eigenmaps \cite{Belkin.2003}\\
\cite{vanderMaaten.2008}\\
t-SNE \cite{vanderMaaten.2008}\\
UMAP \cite{McInnes.09.02.2018}\\
\cite{Inselberg.1985}\\
\cite{Inselberg.2009}\\
\cite{Hoffman.1924Oct.1997}\\
\cite{Daniels.2012} \\
\cite{Kandogan.2000}\\
\cite{RubioSanchez.2016}\\
\cite{Dua.2017}\\
\cite{Lehmann.2013}\\
\cite{Hertwig.2015}\\
\cite{Blum.2008}\\
\cite{Keim.0507July2006}\\
\cite{Keim.0507July2006}\\
\cite{Keim.2010}

% ----------------------------------------------------------
\section{Heuristics}

P.M. Todd defines heuristics as:
\begin{quote}
    '[...]approximate strategies or ‘rules of thumb’ for decision making and problem solving that do not guarantee a correct solution but that typically yield a reasonable solution or bring one closer to hand. As such, they stand in contrast to algorithms that will produce a correct solution given complete and correct inputs.[...]' \cite{P.M..2001}
\end{quote} 


Table \ref{tab:heuvsalgo} shows a comparison between heuristics and complete search:
\begin{table}[H]
\centering
\fbox{%
    \begin{tabular}{ c || c | c }
    & heuristics & complete search\\
    \hline
    \hline
    computation & fast & slow \\
    \hline
    solution & error prone & exact\\
    \hline
    mathematically provable & in most cases no & yes\\ 
    \hline
    based on & intuition, exploration, guesses & finite set of instructions\\
    \end{tabular}
    }%
    \caption{Comparison between heuristics and algorithms}
    \label{tab:heuvsalgo}
\end{table}
According to Ankerst et al. \cite{ANKERST.2000} 

\begin{enumerate}
    \item pattern recognition capabilities of human brain can increase the effectiveness
    \item deeper understanding of the results and thus more trust into the system
    \item domain knowledge by the user can lead to better results and avoid overfitting
\end{enumerate}

Rauber et al. \cite{Rauber.2018} 
Heidari et al. \cite{Heidari.2021} 
Chu et al. \cite{Chu.2006}