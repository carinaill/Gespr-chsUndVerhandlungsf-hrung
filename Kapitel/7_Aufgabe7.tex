{\let\clearpage\relax\vspace{3\baselineskip}
\chapter{Verhandlungstypen erkennen}}
Verhandlungstypen lassen sich nicht anhand formaler Rollen oder persönlicher Sympathien erkennen, sondern vor allem durch das konkrete Verhalten der Beteiligten in der Verhandlungssituation. Entscheidende Hinweise liefern der Umgang mit Positionen und Interessen, die Art der Argumentation (sachlich oder emotional), der Einsatz von Druck oder Beziehung sowie die Bereitschaft zu Zugeständnissen im Umgang mit Konflikten. \\
Das Seminar unterscheidet unter anderem zwischen harten und weichen Verhandlungsstilen, wie sie auch im Harvard-Konzept beschrieben werden. Harte Verhandlungstypen streben Durchsetzung und Kontrolle an und beharren auf Positionen, währen weiche Typen stärker auf Beziehung, Harmonie und Kompromisse fokussieren. \footnote{Vgl. \cite{typenundharvard}} \\
Am Verhandlungstisch lassen sich diese Typen insbesondere anhand des Sprachstils, der Reaktion auf Gegenargumente, des Verhaltens bei Drucksituationen sowie des Umgangs mit Zeit und Entscheidungsprozessen erkennen. Diese Beobachtungen ermöglichen eine realistischere Einschätzung der Verhandlungssituation. \\
Das Erkennen von Verhandlungstypen beeinflusst unmittelbar das eigene Verhalten. Gegenüber harten Typen ist eine sachliche, strukturierte Argumentation sinnvoll, währen emotionale Appelle vermieden werden sollten. Weiche, beziehungsorientierte Typen erfordern hingegen eine stärkere Berücksichtigung der Beziehungsebene und kooperative Lösungsansätze. Ziel ist es nicht, den eigenen Stil aufzugeben, sondern flexibel zu agieren und das Verhalten bewusst an den jeweiligen Verhandlungspartner anzupassen.\footnote{Vgl. \cite{typenundharvard}}
