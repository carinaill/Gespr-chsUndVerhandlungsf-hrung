\chapter{Verhandlungstypen erkennen}
Verhandlungstypen lassen sich nicht anhand formaler Rollen oder persönlicher Sympathien erkennen, sondern vor allem durch das konkrete Verhalten der Beteiligten in der Verhandlungssituation. Der Fokus liegt dabei auf der Art und Weise, wie verhandelt wird. Hinweise liefern insbesondere der Umgang mit Positionen und Interessen, die Form der Argumentation (sachlich vs. Emotional), der Einsatz von Druck oder Beziehung, die Bereitschaft zu Zugeständnissen sowie der Grad an Offenheit im Umgang mit Konflikten und Emotionen.\\
Das Seminar unterscheidet unter anderem zwischen harten und weichen Verhandlungsstilen, wie sie auch im Harvard-Konzept beschrieben werden. Während harte Verhandlungstypen auf Durchsetzung, Kontrolle und das Beharren auf Positionen setzen, legen weiche Verhandlungstypen mehr Wert auf Beziehung, Harmonie und Kompromisse. \footnote{Vgl. \cite{typenundharvard}} \\
Am Verhandlungstisch lassen sich diese Typen vor allem anhand des Sprachstils, der Reaktion auf Gegenargumente, des Verhaltens bei Druck oder Widerstand sowie des Umgangs mit Zeit, Pausen und Entscheidungsprozessen erkennen. Durch diese Beobachtungen kann der Verhandlungstyp besser eingeordnet und die Situation realistischer eingeschätzt werden. \\
Das Erkennen von Verhandlungstypen beeinflusst unmittelbar das eigene Verhalten. Gegenüber harten Verhandlungstypen ist eine sachliche, klare und strukturierte Argumentation sinnvoll, während emotionale Appelle oder voreilige Zugeständnisse vermieden werden sollten. Weiche, beziehungsorientierte Typen erfordern hingegen eine stärkere Berücksichtigung der Beziehungsebene und kooperative Lösungsansätze. Bei zurückhaltenden Typen sind Geduld und aktives Zuhören gefragt, während gegenüber offensiven Typen klare Grenzen bei gleichzeitig ruhigem Auftreten notwendig sind.\\
Es ist nicht das Ziel, den eigenen Stil aufzugeben, sondern flexibel zu agieren und das eigene Verhalten bewusst auf den jeweiligen Verhandlungspartner abzustimmen. Das Seminar betont, dass sachbezogenes und prinzipiengeleitetes Verhandeln langfristig erfolgreicher ist als starres Festhalten an einem bestimmten Stil. \footnote{Vgl. \cite{typenundharvard}}