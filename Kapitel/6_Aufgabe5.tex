\chapter{BATNA und ihre Auswirkungen}
Die systematische Vorbereitung einer BATNA (“Best Alternative to a Negotiated Agreement”) hat einen wesentlichen Einfluss auf das Verhalten und die Entscheidungsmöglichkeiten der Verhandelnden. Eine klar definierte BATNA fungiert als persönliches Sicherheitsnetz und schafft damit emotionale und kognitive Stabilität. Wer seine Alternativen kennt, tritt selbstbewusster auf, wirkt souveräner und kann die eigenen Grenzen klarer formulieren. Dies reduziert Unsicherheit und stärkt die Fähigkeit, auch in schwierigen Gesprächssituationen rationale Entscheidungen zu treffen.\\
Darüber hinaus erhöht eine gut vorbereitete BATNA die strategische Handlungsfähigkeit. Die Verhandelnden wissen, welche Optionen ihnen zur Verfügung stehen, falls keine Einigung erzielt wird, und können dadurch flexibler agieren. Dies verhindert, dass sie sich auf ungünstige Bedingungen einlassen oder aus Angst vor einem Scheitern der Verhandlung zu früh Kompromisse eingehen. Ist das Angebot schlechter als die eigene BATNA, wird ein Abbruch der Verhandlung zur sinnvollen und bewussten Option. \footnote{Vgl. \cite{troczynski2023}} \\
Eine starke BATNA verbessert außerdem die eigene Machtposition. Sie ermöglicht ein selbstbewussteres Auftreten und erleichtert es, bessere Bedingungen auszuhandeln. Umgekehrt kann eine schwache oder unklare BATNA dazu führen, dass die Verhandelnden weniger Druckmittel besitzen und größere Zugeständnisse machen müssen. Die Vorbereitung umfasst daher nicht nur die Identifikation und Bewertung eigener Alternativen, sondern auch die Einschätzung der möglichen BATNA der Gegenseite. Dies erweitert den eigenen Entscheidungsspielraum und unterstützt eine realistische Einschätzung der Verhandlungssituation.\\
Insgesamt führt die Vorbereitung einer BATNA zu mehr Klarheit, Stabilität und strategischer Souveränität. Diese sind zentrale Voraussetzungen für sachgerechte und erfolgreiche Verhandlungen.
