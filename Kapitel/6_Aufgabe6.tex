{\let\clearpage\relax\vspace{3\baselineskip}
\chapter{Analyse einer Verhandlungssituation}}
\section{a: Analyse der Gesprächssituation unter verhandlungstheoretischen Gesichtspunkten}
Eine Verhandlung benötigt mindestens zwei Parteien mit zwei unterschiedlichen Interessen, 
wobei beide ihren Willen bestmöglich durchsetzen wollen und dafür Kompromisse eingehen können.\footnote{Vgl. \cite{verhandlungsführung}, Entwurf einer Definition}
Es besteht eine Abhängigkeit zwischen beiden Parteien, was aus Konfliktpartnern Verhandlungspartner macht.\footnote{Vgl. \cite{verhandlungsführung}, Entwurf einer Definition}
Diese Kriterien sind im vorliegenden Fallbeispiel erfüllt, sodass grundsätzlich von einer Verhandlungssituation 
ausgegangen werden kann.
Ein weiteres zentrales Kennzeichen einer Verhandlungssituation ist allerdings, dass die beteiligten Parteien gemeinsam an einer Lösung arbeiten, 
um ihre Interessen zu verwirklichen.\footnote{Vgl. \cite{verhandlungsführung}, Kennzeichen einer Verhandlungssituation}
Im untersuchten Fall dominieren jedoch argumentativ geführte Wortwechsel, in denen die Vorgesetzten primär ihre Position verteidigen, 
anstatt lösungsorientiert auf die Anliegen der Teamassistentinnen einzugehen. \\
Zwar weist Kreggenfeld darauf hin, dass argumentative Sequenzen Bestandteil vieler Verhandlungen sein können,\footnote{Vgl. \cite{kreggenfeld2021}, S. 9} 
jedoch dienen Argumente dabei der konstruktiven Auseinandersetzung mit den Interessen der Gegenseite. 
Das fehlt hier, da die Aussagen der Teamassistentinnen nicht konstruktiv aufgegriffen, sondern vielmehr umgedeutet werden. 
So interpretiert Herr Köhler strukturelle Informationsprobleme als Kompetenzdefizite der Betroffenen. \\
Der erste Schritt der inhaltlichen Verhandlungsvorbereitung,\footnote{Vgl. \cite{dreischritt}}
die Identifikation und Berücksichtigung relevanter Interessen, wird nicht erfüllt. 
Obwohl die Sachlage nicht eindeutig erscheint, sprechen der bestehende Interessengegensatz im Rahmen der Umstrukturierung, 
der argumentative Aushandlungsprozess sowie die beidseitige Zielorientierung dafür, 
den beschriebenen Austausch als Verhandlung einzuordnen.
Allerdings handelt es sich dabei um eine wenig erfolgreiche Verhandlung, die großes Verbesserungspotenzial hinsichtlich Vorbereitung sowie Kommunikation aufweist.


\section{b: Interessenanalyse der Verhandlungsparteien}
Im Fallbeispiel treten die Teamassistentinnen als Verhandelnde auf, während die Vorgesetzten die Gegenseite bilden. Gegenstand der Verhandlung ist die Aufgabenverteilung nach der Umstrukturierung sowie die Zusammenarbeit mit dem Service Center.
Das ist in Tabelle~\ref{tab:schritt1_feststellung} dargestellt.

\begin{table}[h]
    \centering
    \renewcommand{\arraystretch}{1.3}
    
    \begin{tabularx}{\textwidth}{|c|
                                  >{\raggedright\arraybackslash}X|
                                  >{\raggedright\arraybackslash}X|}
    \hline
    \textbf{Kategorie}
    &
    \multirow{3}{=}{\textbf{1. Schritt:}\\
    Feststellung der relevanten Parteien}
    &
    \textbf{Verhandelnder:} \textit{Teamassistentinnen} \\ \cline{3-3}
    
    &
    &
    \textbf{Gegenseite:} \textit{Vorgesetzte} \\ \cline{3-3}
    
    &
    &
    \textbf{Gegenstand:} Aufgabenverteilung \\
    \hline
    \textbf{Art der Personen}
    &
    \textbf{Personen auf „meiner Seite“, die an dem Ergebnis interessiert sein könnten:}
    &
    \textbf{Personen auf der „Gegenseite“, die an dem Ergebnis interessiert sein könnten:} \\
    \hline
    Interessengruppe
    &
    Teamassistenz-Kolleginnen
    &
    Geschäftsführung / Service Center \\
    \hline
    Freunde
    &
    nicht relevant
    &
    nicht relevant \\
    \hline
    Familie
    &
    indirekt betroffen durch Stress
    &
    nicht relevant \\
    \hline
    Chef
    &
    Projektleiter
    &
    höhere Führungsebene, Geschäftsführung \\
    \hline
    Sonstige
    &
    Kunden, andere Abteilungen die mit der Umstrukturierung unzufrieden sind
    &
    andere Abteilungen, die mit der Umstrukturierung zufrieden sind \\
    \hline
    \end{tabularx}
    \caption{Feststellung der relevanten Parteien}
    \label{tab:schritt1_feststellung}
    \end{table}

Die persönlichen und geschäftlichen Interessen beider Seiten sowie weiterer Beteiligter werden in Tabelle~\ref{tab:schritt2_interessen} identifiziert.

\begin{table}[H]
    \centering
    \renewcommand{\arraystretch}{1.3}
    
    \begin{tabularx}{\textwidth}{|>{\raggedright\arraybackslash}p{0.22\textwidth}|
                                  >{\raggedright\arraybackslash}X|
                                  >{\raggedright\arraybackslash}X|}
    \hline
    &
    \multirow{3}{=}{\textbf{2. Schritt:}\\
    Klärung der Interessen}
    &
    \textbf{Verhandelnder:} \textit{Teamassistentinnen} \\ \cline{3-3}
    
    &
    &
    \textbf{Gegenseite:} \textit{Vorgesetzte} \\ \cline{3-3}
    
    &
    &
    \textbf{Gegenstand:} Aufgabenverteilung \\
    \hline
    \textbf{Meine Interessen}
    &
    \textbf{Interessen der Gegenseite}
    &
    \textbf{Interessen von anderen} \\
    \hline
    \textbf{Persönlich:}
    &
    \textbf{Persönlich:}
    &
    \textbf{Andere 1:} Projektleiter \\
    \hline
    Anerkennung ihrer Arbeitsbelastung
    &
    Autorität bewahren
    &
    korrekte und vollständige Information \\
    \hline
    Workload reduzieren
    &
    als kompetente Führung wahrgenommen werden
    &
    geringe Rückfragen \\
    \hline
    fairer Umgang
    &
    positives Klima (Mahler)
    &
    Pünktlichkeit \\
    \hline
    effiziente Kommunikation
    &
    sichtbarer persönlicher Erfolg
    &
    Rücksicht auf persönliche Arbeitsweise \\
    \hline
    \textbf{Geschäftlich:}
    &
    \textbf{Geschäftlich:}
    &
    \textbf{Andere 2:} Service Center \\
    \hline
    klare Aufgabenverteilung
    &
    Umstrukturierung erfolgreich erscheinen lassen
    &
    klare, standardisierte Information \\
    \hline
    weniger Fehler/Doppelarbeit
    &
    Kostenersparnis (Köhler)
    &
    \textbf{Andere 3:} höhere Führung \\
    \hline
    effizientere Abläufe
    &
    keine Änderung des Systems
    &
    Kostenreduktion \\
    \hline
    direkter Informationsabfluss
    &
    weniger Beschwerden im Team
    &
    Effizienz \\
    \hline
    Entlastung vom Delegationsaufwand
    &
    Standardisierung der Prozesse
    &
    Einheitliche Prozesse \\
    \hline
    \end{tabularx}
    \caption{Klärung der Interessen}
    \label{tab:schritt2_interessen}
\end{table}

In der Tabelle~\ref{tab:schritt3} werden die tieferliegenden Motive betrachtet, die hinter den Interessen stehen. Hier können Modelle wie Maslows Bedürfnispyramide herangezogen werden.
    
\begin{table}[H] 
    \centering 
    \renewcommand{\arraystretch}{1.3} % Zeilenhöhe etwas vergrößern 
    \begin{tabularx}{\textwidth}{|X|p{5cm}|>{\centering\arraybackslash}X|}
    \hline 
    \textbf{3. Schritt: Bedürfnisse und Motive} 
    & 
    \textbf{Tieferliegende Interessen} („Warum?“ und „zu welchem Zweck?“) 
    & 
    \textbf{Relative Bedeutung} (max. 100 Punkte) \\ 
    \hline 
    \textbf{Meine Interessen:} 
    & 
    & \\ 
    \hline 
    Anerkennung 
    & 
    Wertschätzung
    & 80 \\ 
    \hline 
    klare Kommunikation 
    & 
    Struktur, Sicherheit, weniger Stress 
    & 
    100 \\ 
    \hline 
    weniger Fehler 
    & 
    Professionalität, Sicherheit 
    & 
    100 \\ 
    \hline 
    effiziente Prozesse 
    & 
    \makecell[l]{Kompetenz erleben, \\ 
    Wirksamkeit, Planbarkeit} 
    & 
    95 \\ 
    \hline 
    \textbf{Interessen der Gegenseite:} 
    & 
    & \\ 
    \hline 
    keine zusätzliche Kosten 
    & 
    ökonomische Sicherheit 
    & 
    100 \\ 
    \hline 
    Umstrukturierung verteidigen 
    & 
    Geld/Zeit sparen, Autorität 
    & 100 \\ 
    \hline 
    Einheitliche Prozesse 
    & 
    Kontrolle, Ordnung 
    & 
    95 \\ 
    \hline 
    positive Außenwirkung 
    & 
    Reputation 
    & 
    90 
    \\ 
    \hline 
\end{tabularx} 
\caption{Bedürfnisse und Motive} 
\label{tab:schritt3} 
\end{table}

\section{c: Kommunikationsdefizite und ihre Auswirkungen auf den Gesprächsverlauf}
Im Gesprächsverlauf wird deutlich, dass das Verhalten, die Formulierungen sowie die inhaltliche Schwerpunktsetzung
von Frau Schneider und Frau Klar nicht zu einem zielführenden Verlauf beitragen.
Zentrale Kommunikationsprinzipien und grundlegende Elemente erfolgreicher Verhandlungsführung werden dabei verletzt.

\textbf{Punkt 1:} Unterbrechungen und fehlendes aktives Zuhören\\
Der Gesprächsfluss wird mehrfach durch Unterbrechungen gestört. Frau Schneider unterbricht sowohl die Vorgesetzten als auch ihre Kollegin Frau Klar. 
Dadurch wirkt das Gespräch unruhig und wenig kooperativ. 
Dieses Verhalten beeinflusst die Beziehungsebene negativ und steht im Widerspruch zu Watzlawicks zweitem Axiom\footnote{Vgl. \cite{watzlawick_axiome}}, 
da Unterbrechungen als Ungeduld oder mangelnder Respekt wahrgenommen werden können. 
Zudem wird das erste Prinzip des Harvard-Konzepts („Menschen und Probleme getrennt behandeln“) verletzt\footnote{Vgl. \cite{typenundharvard}}, 
da emotionale Reaktionen eine sachliche Lösungsfindung erschweren.

\textbf{Punkt 2:} Emotionale und wenig strukturierte Argumentation\\
Die Teamassistentinnen verwenden teilweise emotional gefärbte und unspezifische Formulierungen, 
etwa „Das kann ich nicht als Kosteneinsparung akzeptieren“ oder „Da kommen nur wieder Rückfragen an mich“. 
Statt klar strukturierter Argumente oder konkreter Forderungen äußern sie vor allem Frustration. 
Dadurch verlagert sich der Fokus auf die Beziehungsebene, was bei den Vorgesetzten zu Rechtfertigung und Abwehr führt. 
Im Sinne von Watzlawicks drittem Axiom entsteht eine Eskalationsdynamik\footnote{Vgl. \cite{watzlawick_axiome}}.

\textbf{Punkt 3:} Fokus auf Symptome statt klare Interessen\\
Frau Schneider und Frau Klar schildern zahlreiche Probleme (wie Informationsverlust oder Mehrarbeit), benennen jedoch keine klaren Interessen oder konkreten Ziele. 
Statt lösungsorientierter Fragen wie „Was soll sich konkret ändern?“ oder „Welche Aufgaben sollten neu geregelt werden?“ verbleibt die Argumentation auf der Symptomebene. 
Dadurch können die Vorgesetzten die Aussagen umdeuten oder relativieren. Zentrale Prinzipien des Harvard-Konzepts, 
insbesondere das klare Benennen von Interessen und das Entwickeln von Optionen, werden somit nicht erfüllt\footnote{Vgl. \cite{typenundharvard}}.

Insgesamt gelingt es den Teamassistentinnen dadurch nicht, ihre Verhandlungsposition klar zu machen oder die Vorgesetzten zu konkreten Zugeständnissen zu bewegen.

\section{Verbesserungsmöglichkeiten der Gesprächs- und Verhandlungsführung}

\subsection*{Teamassistentinnen:}
\textbf{Verbesserung der Vorbereitung}\\
Die Teamassistentinnen hätten sich strukturierter auf das Gespräch vorbereiten müssen. Troczynski betont, dass erfolgreiche Vorbereitung voraussetzt, sich vorab intensiv mit eigenen Interessen, den möglichen Interessen der Gegenseite sowie deren erwartbaren Reaktionen auseinanderzusetzen.\footnote{Vgl. \cite{troczynski2023}, S. 96-99.}  
Statt operative Schwierigkeiten zu schildern, hätten sie  definieren können:
\begin{itemize}[noitemsep, topsep=0pt, leftmargin=*, label=\textbullet]
    \item Welche konkreten Veränderungen sie anstreben,
    \item Welche Prozesse angepasst werden sollen,
    \item Welche Entlastung sie benötigen und warum.
\end{itemize}
Die Unterbrechungen deuten zudem auf fehlende interne Abstimmung hin, die eine konsistente Argumentationslinie erschwert.

\textbf{Verbesserung der Gesprächs- und Verhandlungsführung}\\
Sie hätten stärker auf Sachlichkeit, Struktur und aktives Zuhören achten müssen. Emotionale Formulierungen vermischen Sach- und Beziehungsebene und führen zu Abwehrreaktionen.\footnote{Vgl. \cite{troczynski2023}, S.98, 102-103.}  
Zielführender wäre es gewesen, ruhig zuzuhören, die eigenen Anliegen klar zu strukturieren und lösungsorientierte Beiträge zu formulieren.

\subsection*{Vorgesetzte}
\textbf{Verbesserung der Vorbereitung}\\
Auch die Vorgesetzten hätten sich intensiver vorbereiten müssen. Troczynski betont, dass Führungskräfte vor Gesprächen mit Mitarbeitenden deren Perspektive ernsthaft analysieren sollten.\footnote{Vgl. \cite{troczynski2023}S.96-97, 101.}  
Statt nur die Erfolge der Umstrukturierung hervorzuheben, hätten sie sich Fragen stellen können:
\begin{itemize}[noitemsep, topsep=0pt, leftmargin=*, label=\textbullet]
    \item Welche konkreten Probleme treten in der täglichen Arbeit auf?
    \item Wo besteht echter Anpassungsbedarf?
    \item Welche Punkte sind verhandelbar?
\end{itemize}

\textbf{Verbesserung der Gesprächs- und Verhandlungsführung}\\
Die Vorgesetzten reagieren defensiv und rechtfertigend. Zielführender wäre es gewesen, die Anliegen zunächst anzuerkennen, das Gespräch zu strukturieren und gemeinsame Lösungsoptionen zu entwickeln. Dies entspricht dem dritten Prinzip des Harvard-Konzepts.\footnote{Vgl. \cite{typenundharvard}}

% Kompakte Darstellung der Paragraphen
\setkomafont{paragraph}{\normalfont\bfseries\small}
\RedeclareSectionCommand[
  beforeskip=0.5\baselineskip,
  afterskip=0.25\baselineskip
]{paragraph}

\section{e: Persönlichkeitstypen nach dem Riemann-Thomann-Kreuz}
Im Folgenden werden die beiden Vorgesetzten Herr Köhler und Herr Mahler auf Grundlage ihres Verhaltens im Gespräch tendenziell Persönlichkeitstypen nach dem Riemann-Thomann-Kreuz zugeordnet.\footnote{Vgl. \cite{troczynski2023}, S. 111-112.}

\textbf{Herr Köhler als Distanz -/Dauer orientierter Typ}\\
Herr Köhler tritt sachlich-nüchtern, rational und kontrollierend auf. \footnote{Vgl. \cite{troczynski2023}, S.111, 117-118} Er argumentiert stark aus einer strukturellen und funktionalen Perspektive heraus und zeigt wenig Offenheit für emotionale oder subjektive Aspekte. Auffällig ist, dass er Probleme häufig abstrahiert und verallgemeinert und sehr auf Effizienz und Kosten fokussiert ist. Dieses Verhalten entspricht Merkmalen des Distanz-Typs im Riemann-Thomann-Modell. \footnote{Vgl. \cite{riemann_kreuz}} 
Herr Köhler sollte sachlich, ruhig und klar strukturiert angesprochen werden. Argumente sollten faktenbasiert, logisch aufgebaut und anhand von Beispielen erläutert werden. Emotionale Vorwürfe sind zu vermeiden, besser sind konkrete Lösungsvorschläge.

\textbf{Herr Mahler als Nähe- / Dauer-orientierter Typ}\\
Herr Mahler zeigt sich grundsätzlich zugewandt und verständnisvoll. Er signalisiert, dass ihm die Probleme bekannt sind, versucht jedoch Spannungen zu relativieren und vermeidet klare Entscheidungen. Stattdessen verweist er auf Entwicklungsprozesse und Teamdynamiken. Dieses beziehungsorientierte Verhalten entspricht Merkmalen des Nähe-Typs. \footnote{Vgl. \cite{riemann_kreuz}}
Herr Mahler sollte wertschätzend und respektvoll angesprochen werden, wobei die Beziehungsebene aktiv berücksichtig werden sollte. Es ist sinnvoll, Verständnis zu signalisieren und gemeinsame Ziele zu betonen, während Anliegen klar, ohne konfrontativen Ton formuliert werden.\\
Die Teamassistentinnen sollten sich vor dem Gespräch bewusst machen, welcher Typ ihnen gegenübersitzt, welche Kommunikationsform dieser bevorzugt und wie sie ihre Inhalte entsprechend übersetzen können.
In diesem konkreten Fallbeispiel hätten sie Argumente als Kombination aus Lösungsvorschlägen sowie menschlicher Wirkung formulieren sollen. 
Das Riemann-Thomann-Modell verdeutlicht, dass gleiche Inhalte je nach Persönlichkeitstyp unterschiedlich vermittelt werden müssen, um wirksam zu sein. \footnote{Vgl. \cite{troczynski2023}, S. 111, 119.}