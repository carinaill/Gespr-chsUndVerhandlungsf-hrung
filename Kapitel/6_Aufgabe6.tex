{\let\clearpage\relax\vspace{3\baselineskip}
\chapter{Analyse einer Verhandlungssituation}}
\section{a: Analyse der Gesprächssituation unter verhandlungstheoretischen Gesichtspunkten}
Im Fallbeispiel findet keine Verhandlung im eigentlichen Sinne statt. Eine Verhandlung setzt das gemeinsame Ziel einer verbindlichen Einigung, den Austausch von Angeboten sowie eine systematische Interessenklärung voraus. Es dominiert jedoch argumentativ geführte Wortwechsel, in denen die Vorgesetzten ihre Position verteidigen, anstatt lösungsorientiert auf die Anliegen der Teamassistentinnen einzugehen. \\
Zwar weist Kreggenfeld darauf hin, dass argumentative Sequenzen Bestandteil vieler Verhandlungen sein können, . \footnote{Vgl. \cite[S. 7]{kreggenfeld2021}} doch dienen Argumente dort der konstruktiven Auseinandersetzung mit den Interessen der Gegenseite. Dies fehlt hier, da die Aussagen der Teamassistentinnen nicht validiert, sondern umgedeutet werden, etwa wenn Herr Köhler strukturelle Informationsprobleme als Kompetenzfrage interpretiert. \\
Bereits der erste Schritt der inhaltlichen Verhandlungsvorbereitung, \footnote{Vgl. \cite{dreischritt}} die Identifikation und Berücksichtigung relevanter Interessen, wird nicht erfüllt. Zwar sind die relevanten Akteure anwesend, doch verhindern Unterbrechungen und fehlendes aktives Zuhören eine klare Interessenartikulation. \\
Da sowohl zentrale strukturelle Merkmale als auch kommunikative Voraussetzungen einer Verhandlung fehlen, handelt es sich nicht um eine echte Verhandlung, sondern um eine argumentativ geprägte Diskussion.

\section{b: Interessenanalyse der Verhandlungsparteien}
Beim Schritt 1 treten im Fallbeispiel die Teamassistentinnen als Verhandelnde auf, während die Vorgesetzten die Gegenseite bilden. Gegenstand der potenziellen Verhandlung ist die Aufgabenverteilung nach der Umstrukturierung sowie die Zusammenarbeit mit dem Service Center.

Beim Schritt 2 werden die persönlichen und geschäftlichen Interessen beider Seiten analysiert sowie die Interessen weiterer Beteiligter identifiziert.

Beim Schritt 3 werden die tieferliegenden Motive betrachtet, die hinter den Interessen stehen. Hier können Modelle wie Maslows Bedürfnispyramide herangezogen werden.

\section{c: Kommunikationsdefizite und ihre Auswirkungen auf den Gesprächsverlauf}
Im Gesprächsverlauf wird deutlich, dass das Verhalten, die Formulierungen und die inhaltliche Schwerpunktsetzung der Teamassistentinnen nicht zu einem zielführenden Verlauf beitragen. Sowohl grundlegende Kommunikationsprinzipien als auch zentrale Elemente erfolgreicher Verhandlungsführung werden verletzt.

Punkt 1: Häufige Unterbrechungen und fehlendes aktives Zuhören\\
Der Gesprächsfluss wird mehrfach durch Unterbrechungen gestört. Frau Schneider unterbricht Herr Mahler und ihre Kollegin Frau Klar, obwohl beide die gleiche Position haben. Dieses Verhalten widerspricht Watzlawicks Axiom 2, \footnote{Vgl. \cite{watzlawick_axiome}} da durch die Unterbrechungen die Beziehungsebene negativ beeinflusst wird, die Teamassistentinnen erscheinen ungeduldig und wenig kooperativ, während die Vorgesetzten dominant wirken. Zudem wird das Prinzip 1 des Harvard-Konzepts („Menschen und Probleme getrennt behandeln“) \footnote{Vgl. \cite{typenundharvard}} verletzt. Die Assistentinnen reagieren auf Aussagen der Vorgesetzten nicht sachlich, sondern impulsiv und auf der Beziehungsebene, wodurch die Gegenseite in eine Verteidigungshaltung gerät.

Punkt 2: Emotional gefärbte Formulierungen und fehlende Struktur in der Darstellung\\
Frau Schneider formuliert emotional gefärbte Aussagen, wie „Das kann ich nicht als Kosteneinsparung akzeptieren.“, während Frau Klar unstrukturierte Problembeschreibungen formuliert,“bevor ich es lange erklären muss…. Da kommen nur wieder Rückfragen an mich.” Statt klarer Interessen oder Forderungen werden überwiegend Beschwerden geäußert. Die stark emotionale Kommunikation verstärkt den Beziehungsaspekt (Axiom 2). Die Vorgesetzten reagieren darauf mit Rechtfertigung und Abwehr, wodurch eine Eskalationsspirale im Sinne von Axiom 3\footnote{Vgl. \cite{watzlawick_axiome}} entsteht.

Punkt 3: Fokussierung auf Symptome statt strategischer Interessen\\
Die Assistentinnen sprechen vor allem über Probleme wie Informationsverlust, Arbeitsüberlastung, Rückfragen des Service Centers oder unterschiedliche Arbeitsweisen der Projektleiter. Aber sie benennen keinen klaren Interessen oder Ziele. Hier wäre es hilfreich zu fragen: “Was soll sich konkret ändern?”, “Welche Struktur wäre hilfreich?”, “Welche Prozesse sollten angepasst werden?”.\footnote{Vgl. \cite{typenundharvard}} Die Vorgesetzten greifen diese Darstellungen auf und deuten sie zum Teil um. Ein Beispiel ist Herr Köhler, der Frau Klars strukturellen Hinweis fälschlicherweise als Kompetenzproblem interpretiert (“…es sollte kein Problem sein, intern zu koordinieren”).
Beim Harvard Konzept werden Prinzip 2 und auch Prinzip 3 (Entwickeln Sie Optionen, bevor Sie entscheiden)\footnote{Vgl. \cite{typenundharvard}} verletzt. Weder werden Interessen klar geäußert, noch entstehen Optionen.

\section{d: Verbesserungsmöglichkeiten der Gesprächs- und Verhandlungsführung}
Die Verbesserungsmöglichkeiten der Teamassistentinnen:
\begin{itemize}
    \item Verbesserung der Vorbereitung:
\end{itemize} 
Die Teamassistentinnen hätten sich strukturierter auf das Gespräch vorbereiten müssen. 
Troczynski betont, dass eine erfolgreiche Verhandlungsvorbereitung voraussetzt, sich vorab intensiv mit den eigenen Interessen, den möglichen Interessen der Gegenseite, 
sowie mit deren erwartbaren Reaktionen auseinandersetzen. 
Statt überwiegend operative Schwierigkeiten zu schildern, hätten sie beispielweise definieren können, „Welche konkreten Veränderungen sie anstreben“, „Welche Prozesse angepasst werden sollen“, „Welche Entlastung sie benötigen und warum“. \footnote{Vgl. \cite[S.92]{troczynski2023}} 
Die gegenseitigen Unterbrechungen deuten zu dem auf eine fehlende interne Abstimmung hin, die eine konsistente Argumentationslinie erschwert.
\begin{itemize}
    \item Verbesserung der Gesprächs - und Verhandlungsführung:
\end{itemize} 
In der Gesprächsführung hätten die Teamassistentinnen stärker auf Sachlichkeit, Struktur und aktives Zuhören achten müssen. Emotionale Formulierungen vermischen Sach- und Beziehungsebene und führen zu Abwehrreaktionen. Zielführender wäre, ruhig zuzuhören, die eigenen Anliegen klar zu strukturieren und bewusst lösungsorientierte Beiträge zu formulieren.

Die Verbesserungsmöglichkeiten der Vorgesetzten: 
\begin{itemize}
    \item Verbesserung der Vorbereitung:
\end{itemize} 
Die Vorgesetzten hätten sich auch intensiver vorbereiten müssen. Troczynski betont, dass Führungskräfte vor Gesprächen mit Mitarbeitende deren Perspektive ernsthaft analysieren sollten. Anstatt in erster Linie die Erfolge der Umstrukturierung hervorzuheben, hätten sie sich vorbereitende Fragen stellen können, „Welche konkreten Probleme treten in der täglichen Arbeit auf?“, „Wo könnte ein echter Anpassungsbedarf bestehen?“, „Welche Punkte sind verhandelbar?“. \footnote{Vgl. \cite[S.102]{troczynski2023}} 
\begin{itemize}
    \item Verbesserung der Gesprächs - und Verhandlungsführung:
\end{itemize} 
In der Gesprächsführung reagieren die Vorgesetzte defensiv und rechtfertigend. Zielführender wäre es gewesen, die Anliegen zunächst anzuerkennen, das Gespräch zu strukturieren und gemeinsame Lösungsoptionen zu entwickeln. Dies entspricht dem dritten Prinzip des Harvard-Konzepts, das die Entwicklung von Optionen vor Entscheidungen fordert. \footnote{Vgl. \cite{typenundharvard}}

\section{e: Persönlichkeitstypen nach dem Riemann-Thomann-Kreuz}
Im Folgenden werden die beiden Vorgesetzten Herr Köhler und Herr Mahler auf Grundlage ihres Verhaltens im Gespräch tendenziell Persönlichkeitstypen nach dem Riemann-Thomann-Kreuz zugeordnet.
\begin{itemize}
    \item Herr Köhler (Distanz -/Dauer orientierter Typ) 
\end{itemize} 
Herr Köhler tritt sachlich-nüchtern, rational und kontrollierend auf. \footnote{Vgl. \cite[S.111]{troczynski2023}} Er argumentiert stark aus einer strukturellen und funktionalen Perspektive heraus und zeigt wenig Offenheit für emotionale oder subjektive Aspekte. Auffällig ist, dass er Probleme häufig abstrahiert und verallgemeinert, etwa wenn er die geschilderten Schwierigkeiten auf die Art der Informationsweitergabe reduziert oder implizit als Kompetenzfrage der Teamassistentinnen umdeutet. Dieses Verhalten entspricht Merkmalen des Distanz-Typs im Riemann-Thomann-Modell. \footnote{Vgl. \cite{riemann_kreuz}} 
Herr Köhler sollte sachlich, ruhig und klar strukturiert angesprochen werden. Argumente sollten faktenbasiert, logisch aufgebaut und anhand konkreter Beispiele erläutert werden. Emotionale Vorwürfe sind vermeiden.
\begin{itemize}
    \item Herr Mahler (Nähe- / Dauer-orientierter Typ) 
\end{itemize} 
Herr Mahler zeigt sich grundsätzlich zugewandt und verständnisvoll. \footnote{Vgl. \cite[S.111]{troczynski2023}} Er signalisiert, dass ihm die Probleme bekannt sind, versucht jedoch Spannungen zu relativieren und vermeidet klare Entscheidungen. Stattdessen verweist er auf Entwicklungsprozesse, Gewöhnung und Teambuilding. Dieses Verhalten entspricht Merkmalen des Nähe-Typs. \footnote{Vgl. \cite{riemann_kreuz}}
Herr Mahler sollte wertschätzend und respektvoll angesprochen werden, wobei die Beziehungsebene aktiv berücksichtig werden sollte. Es ist sinnvoll, Verständnis zu signalisieren und gemeinsame Ziele zu betonen, während Anliegen klar, ohne konfrontativen Ton formuliert werden.\\
Die Teamassistentinnen sollten sich vor dem Gespräch bewusst machen, welcher Typ ihnen gegenübersitzt, welche Kommunikationsform dieser bevorzugt, wie sie ihre Inhalte entsprechen übersetzen können. Das Riemann-Thomann-Modell verdeutlicht, dass gleiche Inhalte je nach Persönlichkeitstyp unterschiedlich vermittelt werden müssen, um wirksam zu sein.