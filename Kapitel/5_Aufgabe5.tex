{\let\clearpage\relax\vspace{3\baselineskip}
\chapter{BATNA und ihre Auswirkungen}}
Die systematische Vorbereitung einer BATNA („Best Alternative to a Negotiated Agreement“) hat einen erheblichen Einfluss auf das Verhalten und die Entscheidungsmöglichkeiten der Verhandelnden. Eine klar definierte BATNA wirkt als persönliches Sicherheitsnetz, stärkt Selbstbewusstsein und Souveränität und ermöglicht es, eigene Grenzen klar zu formulieren. Dadurch werden Unsicherheit reduziert und rationale Entscheidungen auch in schwierigen Verhandlungssituationen erleichtert. \\
Zugleich erhöht eine gut vorbereitete BATNA die strategische Handlungsfähigkeit. Wer seine Alternativen kennt, kann flexibler agieren und vermeidet es, aus Angst vor dem Scheitern der Verhandlung ungünstigste Kompromisse einzugehen. Ist ein Angebot schlechter als die eigene BATNA, wird ein Abbruch der Verhandlung zu einer bewussten und sinnvollen Option. \footnote{Vgl. \cite [S. 133]{troczynski2023}} \\
Darüber hinaus stärkt eine starke BATNA die eigene Verhandlungsposition, während eine schwache oder unklare BATNA zu erhöhten Zugeständnissen führen kann. Die Vorbereitung umfasst daher nicht nur die Bewertung eigener Alternativen, sondern auch die Einschätzung der möglichen BATNA der Gegenseite, um die Verhandlungssituation realistisch einzuordnen. 
